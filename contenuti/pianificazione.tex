\chapter{Pianificazione}
\chapter{Pianificazione}
La pianificazione è il processo di riflessione sulle attività necessarie per raggiungere un obiettivo desiderato. È la prima e più importante attività per ottenere i risultati desiderati. Implica la creazione e il mantenimento di un piano, come gli aspetti psicologici che richiedono abilità concettuali, come l'uso della logica e dell'immaginazione per visualizzare non solo un risultato finale desiderato, ma i passaggi necessari per ottenere quel risultato. In questo progetto si è scelto di suddividere la pianificazione in 4 fasi: 
\begin{itemize}
    \item Analisi;
    \item Produzione del Proof of Concept;
    \item Progettazione di dettaglio e codifica;
    \item Validazione e collaudo.
\end{itemize}
Essendo il ciclo di vita di un SW rappresentabile come un automa a stati finiti, gli stati saranno le fasi sopra elencate, mentre gli archi saranno rappresentati da tutte le attività che verranno svolte per passare da uno stato all'altro. Di ogni fase andremo inoltre a specificare quando iniziare (pre-condizioni) e quando finire (post-condizioni).

\section{Analisi}

\textbf{Inizio:} 2022 - 11 - 07 

\textbf{Fine:} 2023 - 0x - xx


Pre-condizioni:
\begin{itemize}
    \item Formazione gruppo;
    \item presentazione capitolato.
\end{itemize}

Post-condizioni:
    \begin{itemize}
        \item Determinazione di un way of working interno al gruppo;
        \item Redazione dei documenti:
        \begin{itemize}
            \item analisi dei requisiti;
            \item norme di progetto;
            \item piano di progetto;
            \item piano di qualifica;
            \item glossario.
        \end{itemize}
        \item Verifica dei documenti redatti;
    \end{itemize}

\subsection{Ruoli Attivi}
\begin{itemize}
    \item Responsabile
    \item Amministratore
    \item Analista
    \item Verificatore
\end{itemize}

\subsection{Periodi e Attività}

\subsubsection{Primo Periodo}
Questo periodo va \textbf{dal 2022-11-07 al 2022-11-28} durante il quale si è scelto il capitolato di maggior interesse sulla base di discussioni interne, nelle quali ogni membro ha esposto i propri dubbi e interessi. In seguito all'assegnazione del capitolato scelto si è iniziato lo studio di fattibilità del progetto. 
In primis sono state prese decisioni quali: il nome del team, il logo, l’indirizzo email di riferimento, la frequenza degli incontri, gli strumenti per la comunicazione tra i vari membri.
Sono stati definiti i template per la documentazione, in particolare è stata fatta una prima stesura dei casi d'uso.

\subsubsection{Secondo Periodo}
Questo periodo va \textbf{dal 2022-11-29 al 2022-12-19} durante il quale è stata iniziata la stesura del glossario, utile al chiarimento di terminologie specifiche, il piano di progetto, contenente l’esposizione della pianificazione del lavoro da svolgere nel corso del progetto, e le norme di progetto per fissare le regole base delle attività del gruppo. 
Inoltre si è proseguita l'analisi dei requisiti dove sono stati corretti gli errori segnalati dal professor Cardin e si sono svolte delle attività di formazione delle varie tecnologie, in seguito a una discussione con il proponente che dovranno essere utilizzati per lo sviluppo del Proof of Concept e dell’intero progetto.

\subsubsection{Terzo Periodo}
Questo periodo va \textbf{dal 2022-12-20 al 2023-0x-xx} durante il quale è stato completato il documento del Proof of Concept, è iniziata la stesura del piano di qualifica, è continuata l'analisi dei requisiti, con alcuni UC che sono stati rivisti a causa di problemi relativi alla fornitura di hardware da parte del proponente, ed è avanzata la stesura del piano di progetto e delle norme di progetto.
Sono iniziate le prime attività di verifica incrementale per i documenti in corso di stesura atti alla candidatura alla consegna per l' RTB.

\subsubsection{Quarto Periodo}
Questo periodo va \textbf{dal 2023-xx-xx al 2023-xx-xx} durante il quale sono statte effettuate verifiche di coerenza e coesione complessiva dei documenti. Come stabilito nelle Norme di Progetto tutti i documenti sono stati resi conformi e nel glossario sono stati aggiunti i termini mancanti.

(Inserire immagine Diagramma Gantt)

\section{Produzione del Proof of Concept}

\subsection{Primo Periodo}

\subsection{Secondo Periodo}

\subsection{Terzo Periodo}

(Inserire Diagramma Gantt)