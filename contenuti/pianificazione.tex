\chapter{Pianificazione}
La pianificazione è il processo di riflessione sulle attività necessarie per raggiungere un obiettivo desiderato. È la prima e più importante attività per ottenere i risultati desiderati. Implica la creazione e il mantenimento di un piano, come gli aspetti psicologici che richiedono abilità concettuali, come l'uso della logica e dell'immaginazione per visualizzare non solo un risultato finale desiderato, ma i passaggi necessari per ottenere quel risultato. In questo progetto si è scelto di suddividere la pianificazione in 4 fasi: 
\begin{itemize}
    \item Analisi;
    \item Produzione del Proof of Concept;
    \item Progettazione di dettaglio e codifica;
    \item Validazione e collaudo.
\end{itemize}
Essendo il ciclo di vita di un SW rappresentabile come un automa a stati finiti, gli stati saranno le fasi sopra elencate, mentre gli archi saranno rappresentati da tutte le attività che verranno svolte per passare da uno stato all'altro. Di ogni fase andremo inoltre a specificare quando iniziare (pre-condizioni) e quando finire (post-condizioni).

\section{Analisi}

\textbf{Inizio:} 2022 - 11 - 07 

\textbf{Fine:} 2023 - 0x - xx


Pre-condizioni:
\begin{itemize}
    \item Formazione gruppo;
    \item presentazione capitolato.
\end{itemize}

Post-condizioni:
    \begin{itemize}
        \item Determinazione di un way of working interno al gruppo;
        \item Redazione dei documenti:
        \begin{itemize}
            \item analisi dei requisiti;
            \item norme di progetto;
            \item piano di progetto;
            \item piano di qualifica;
            \item glossario.
        \end{itemize}
        \item Verifica dei documenti redatti;
    \end{itemize}

\subsection{Ruoli Attivi}
\begin{itemize}
    \item Responsabile
    \item Amministratore
    \item Analista
    \item Verificatore
\end{itemize}

\subsection{Periodi e Attività}

\subsubsection{Primo Periodo}
Questo periodo va \textbf{dal 2022-11-07 al 2022-11-28} durante il quale si sceglierà il capitolato di maggior interesse sulla base di discussioni interne, nelle quali ogni membro esporrà i propri dubbi e interessi. In seguito all'assegnazione del capitolato scelto potrà iniziare lo studio di fattibilità del progetto. 
Le prime decisioni collettive che verranno prese saranno: il nome del team, il logo, l’indirizzo email di riferimento, la frequenza degli incontri, gli strumenti per la comunicazione tra i vari membri.
Andranno inoltre definiti i template per la documentazione.

\subsubsection{Secondo Periodo}
Questo periodo va \textbf{dal 2022-11-29 al 2022-12-18} ed è il periodo più ricco in quanto inizieranno le stesure dei principali documenti del progetto. Durante questo periodo verranno redatti il glossario, utile al chiarimento di terminologie specifiche, il piano di progetto, contenente l’esposizione della pianificazione del lavoro da svolgere nel corso del progetto, e le norme di progetto per fissare le regole base delle attività del gruppo. 
In parallelo proseguirà l'analisi dei requisiti dove andranno corretti eventuali errori che verranno segnalati arrivando a stendere una versione rudimentale dell'analisi dei requisiti che verrà successivamente raffinata.

\subsubsection{Terzo Periodo}
Questo periodo va \textbf{dal 2022-12-19 al 2023-01-22} durante il quale inizieranno le stesure di {\it{Proof of Concept}} e piano di qualifica. In parallelo proseguiranno l'analisi dei requisiti, con alcuni UC che andranno riviste a causa di problemi relativi alla fornitura di hardware da parte del proponente, il piano di progetto e le norme di progetto.
Si prevede inoltre di svolgere le prime attività di verifica incrementale per i documenti in corso di stesura atti alla candidatura alla consegna per l'RTB.

\subsubsection{Quarto Periodo}
Questo periodo va \textbf{dal 2023-xx-xx al 2023-xx-xx} durante il quale sono state effettuate verifiche di coerenza e coesione complessiva dei documenti. Come stabilito nelle Norme di Progetto tutti i documenti sono stati resi conformi e nel glossario sono stati aggiunti i termini mancanti.

(Inserire immagine Diagramma Gantt)

\section{Produzione del {\it{Proof of Concept}}}

\textbf{Inizio:} 2023 - 01 - 01 

\textbf{Fine:} 2023 - 0x - xx

Pre-condizioni:
\begin{itemize}
    \item Non sono previste pre-condizioni, queste due fasi verranno svolte quasi in sovrapposizione;
\end{itemize}

Post-condizioni:
    \begin{itemize}
        \item Aggiornamento e approvazione dei documenti prodotti nella fase precedente;
        \item produzione del {\it{Proof of Concept}} e relativa documentazione;
        \item produzione della presentazione per la {\it{Requirement and Technology Baseline}};
    \end{itemize}

\subsection{Ruoli Attivi}
\begin{itemize}
    \item Responsabile
    \item Amministratore
    \item Analista
    \item Verificatore
\end{itemize}

\subsection{Periodi e Attività}

\subsubsection{Primo Periodo}
Questo periodo va \textbf{dal 2023-01-23 al 2023-02-12} durante il quale si prevede di completare e verificare, sia con il proponente che con il docente, il {\it{Proof of Concept}} e l'analisi dei requisiti.
Proseguirà la preparazione dei documenti necessari per la candidatura alla consegna per l'RTB con l'obiettivo di arrivare a definire una data di consegna.

\subsubsection{Secondo Periodo}

\subsubsection{Terzo Periodo}

(Inserire Diagramma Gantt)