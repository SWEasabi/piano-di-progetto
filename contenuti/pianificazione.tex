\chapter{Pianificazione}
La pianificazione è il processo di riflessione sulle attività necessarie per raggiungere un obiettivo desiderato. È la prima e più importante attività per ottenere i risultati desiderati. Implica la creazione e il mantenimento di un piano, come gli aspetti psicologici che richiedono abilità concettuali, come l'uso della logica e dell'immaginazione per visualizzare non solo un risultato finale desiderato, ma i passaggi necessari per ottenere quel risultato. In questo progetto si è scelto di suddividere la pianificazione in 6 fasi: 
\begin{itemize}
    \item Fase di analisi preliminare;
    \item Progettazione Technology Baseline;
    \item Proof of Concept;
    \item Progettazione e codifica requisiti obbligatori;
    \item Progettazione e codifica requisiti opzionali;
    \item Validazione e Test.
\end{itemize}
Essendo il ciclo di vita di un SW rappresentabile come un automa a stati finiti, gli stati saranno le fasi sopra elencate, mentre gli archi saranno rappresentati da tutte le attività che verranno svolte per passare da uno stato all'altro. Di ogni fase andremo inoltre a specificare quando iniziare (pre-condizioni) e quando finire (post-condizioni).
Inoltre, per ogni fase verranno pianificati un preventivo costi e la suddivisione dei ruoli.

\textbf{Suddivisione ruoli:}
Durante la realizzazione di ogni documento, i ruoli saranno assegnati in base alla disponibilità di tempo di ogni membro del gruppo. Ciò significa che un membro del gruppo può assumere lo stesso ruolo per documenti diversi, ma sarebbe responsabile soltanto di un documento alla volta. I responsabili verranno cambiati all'inizio di ogni baseline. Questa stessa suddivisione verrà applicata durante la stesura dell'architettura e l'implementazione del software.

Al termine del progetto didattico, ogni membro del gruppo avrà assunto almeno una volta tutti i ruoli.

\section{Fase di analisi preliminare}

\textbf{Periodo di svolgimento}
Data di inizio: 2022 - 11 - 07 
Data di fine: 2023 - 0x - xx

\textbf{Pre-condizioni:}
\begin{itemize}
    \item Formazione gruppo;
    \item presentazione capitolato.
\end{itemize}

\textbf{Post-condizioni:}
    \begin{itemize}
        \item Determinazione di un way of working interno al gruppo;
        \item Redazione dei documenti:
        \begin{itemize}
            \item analisi dei requisiti;
            \item norme di progetto;
            \item piano di progetto;
            \item piano di qualifica;
            \item glossario.
        \end{itemize}
        \item Verifica dei documenti redatti;
    \end{itemize}

\subsection{Ruoli attivi}
\begin{itemize}
    \item Responsabile
    \item Amministratore
    \item Analista
    \item Verificatore
\end{itemize}

\subsection{Preventivo costi}

\section{Progettazione Technology Baseline}

\section{Proof of Concept};

\section{Progettazione e codifica requisiti obbligatori}

\section{Progettazione e codifica requisiti opzionali}

\section{Validazione e Test}