\chapter{Pianificazione}
La pianificazione è il processo di riflessione sulle attività necessarie per raggiungere un obiettivo desiderato. È la prima e più importante attività per ottenere i risultati desiderati. Implica la creazione e il mantenimento di un piano, come gli aspetti psicologici che richiedono abilità concettuali, come l'uso della logica e dell'immaginazione per visualizzare non solo un risultato finale desiderato, ma i passaggi necessari per ottenere quel risultato. In questo progetto si è scelto di suddividere la pianificazione in 4 fasi: 
\begin{itemize}
    \item Analisi dei rischi;
    \item Produzione del Proof of Concept;
    \item Progettazione di dettaglio e codifica;
    \item Validazione e collaudo.
\end{itemize}
Essendo il ciclo di vita di un SW rappresentabile come un automa a stati finiti, gli stati saranno le fasi sopra elencate, mentre gli archi saranno rappresentati da tutte le attività che verranno svolte per passare da uno stato all'altro. Di ogni fase andremo inoltre a specificare quando iniziare (pre-condizioni) e quando finire (post-condizioni).

\section{Analisi dei rischi}

\section{Produzione del Proof of Concept}