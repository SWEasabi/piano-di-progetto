\chapter{Analisi dei rischi}
Nel corso dello sviluppo del progetto è naturale incontrare vari tipi di problematiche, che con un’attenta e continua analisi dei rischi possono essere mitigate. Il piano per la gestione dei rischi viene suddiviso in 4 attività:
\begin{itemize}
    \item individuazione dei possibili eventi che possono portare a dei problemi durante l’avanzamento;
    \item analisi del problema, in particolare la probabilità con cui si possa verificare e le conseguenze
    negative che comporta;
    \item pianificazione di misure da prendere per impedire il verificarsi dei rischi e comportamenti da seguire nel caso in cui essi dovessero presentarsi. In questo modo si evita che un rischio possa diventare insostenibile;
    \item monitoraggio continuo dei rischi, cercando di prevenirli o minimizzando l’effetto negativo di quest’ultimi.
\end{itemize}

\section{Rischi tecnologici}

\section{Rischi personali}

\section{Rischi organizzativi}

\section{Rischi legati ai requisiti}
