\documentclass{report}
\usepackage[utf8]{inputenc}
\usepackage[T1]{fontenc}
\usepackage{CormorantGaramond}
\usepackage{fontspec}
\usepackage{geometry}
\usepackage[table]{xcolor}
\usepackage{tabularx}
\usepackage{graphicx}
\usepackage{mathtools}
\usepackage[bottom]{footmisc}
\usepackage[italian]{babel}
\usepackage{hyperref}
\usepackage{titlesec}
\usepackage{listings}
\usepackage{color}
\usepackage{graphicx}
\usepackage{fancyhdr}
\usepackage{pgf-pie}
\usepackage{float}

\renewcommand{\headrulewidth}{0.4pt}
\renewcommand{\footrulewidth}{0.4pt}

\lstset{ % General setup for the package
    basicstyle=\small\sffamily,
    numbers=left,
     numberstyle=\tiny,
    frame=tb,
    tabsize=4,
    columns=fixed,
    showstringspaces=false,
    showtabs=false,
    keepspaces,
    commentstyle=\color{red},
    keywordstyle=\color{blue}
}

\geometry{
a4paper,
total = {170mm, 240mm},
left = 20mm,
top = 20mm,
}

\setlength{\headheight}{33.60004pt}

\setlength{\parindent}{0em}
\setlength{\parskip}{0.7em}

\titlespacing{\section}{0pt}{0.7em}{0.5em}
\titlespacing{\subsection}{0pt}{0.7em}{0.5em}

\newcommand{\gassets}{../}

\renewcommand{\title}{
    Piano di Progetto
    
    \tiny Versione documento: \textit{V1.0.1}
}

\newcommand{\people}{
    \normalsize
    \begin{center}
        \begin{tabularx}{7cm}{l | X}            
            \textbf{Uso} & Esterno\\
            \textbf{Destinatario} & Committente\\
            & Cliente \\
        \end{tabularx}
    \end{center}
}

\fancypagestyle{plain}{%
    \fancyhead{} % clear all header fields
    \fancyhead[L]{\leftmark}
    \fancyhead[R]{\textit{SWEasabi} \includegraphics[height=30pt]{\gassets global-assets/img/loghi/SWEasabi_compact_logo.png}}
    \fancyfoot{} % clear all footer fields
    \fancyfoot[L]{\thepage}
    \fancyfoot[R]{Piano di Progetto}
}

\begin{document}

\pagestyle{fancy}

\fancyhead{} % clear all header fields
\fancyhead[L]{\leftmark}
\fancyhead[R]{\textit{SWEasabi} \includegraphics[height=30pt]{\gassets global-assets/img/loghi/SWEasabi_compact_logo.png}}
\fancyfoot{} % clear all footer fields
\fancyfoot[L]{\thepage}
\fancyfoot[R]{Piano di progetto}


\input{\gassets global-assets/tex/header}
\thispagestyle{empty}
\clearpage
\pagenumbering{Roman}
\section{Registro delle modifiche}

\newcommand{\pzerozerouno}{
    \begin{center}
        \begin{tabularx}{\linewidth}{l | X}
            \textbf{Approvazione} & \
            \hline
            \textbf{Redazione}& Peron Samuel\
            & Romano Davide\
            \hline
            \textbf{Verifica} & \\
        \end{tabularx}
    \end{center}
}
\newcommand{\mzerozerouno}{
    \begin{itemize}
        \item Aggiunta sezione introduzione
        \item Aggiunta sezione analisi dei rischi
        \item Aggiunta sezione modello
        \item Aggiunta sezione pianificazione
        \item Aggiunta sezione preventivo
    \end{itemize}
}

\newcommand{\vzerozerouno}{
    \hline
    0.0.1 & 5 gen 2023 & \mzerozerouno & \pzerozerouno\
}






\newcommand{\pzerozerodue}{
    \begin{center}
        \begin{tabularx}{\linewidth}{l | X}
            \textbf{Approvazione} & Massarenti Alessandro\\
            \hline
            \textbf{Redazione}& Peron Samuel\\
            & Romano Davide\\
            \hline
            \textbf{Verifica} & Pierobon Luca\\
            & Massarenti Alessandro\\
            & Bonavigo Michele \\
        \end{tabularx}
    \end{center}
}
\newcommand{\mzerozerodue}{
    \begin{itemize}
        \item Aggiunto paragrafo relativo all'introduzione
        \item Aggiunto paragrafo relativo al modello di sviluppo adottato
    \end{itemize}
}

\newcommand{\vzerozerodue}{
    \hline
    0.0.2 & 7 gen 2023 & \mzerozerodue & \pzerozerodue\\
}
\newcommand{\pzerounozero}{
    \begin{center}
        \begin{tabularx}{\linewidth}{l | X}
            \textbf{Approvazione} & Massarenti Alessandro\\
            \hline
            \textbf{Redazione}& Peron Samuel\\
            & Romano Davide\\
            \hline
            \textbf{Verifica} & Pierobon Luca\\
            & Massarenti Alessandro\\
            & Bonavigo Michele \\
        \end{tabularx}
    \end{center}
}
\newcommand{\mzerounozero}{
    \begin{itemize}
        \item Verifica generale del documento
    \end{itemize}
}

\newcommand{\vzerozerouno}{
    \hline
    0.1.0 & 7 gen 2023 & \mzerounozero & \pzerounozero\\
}
\newcommand{\pzerounouno}{
    \begin{center}
        \begin{tabularx}{\linewidth}{l | X}
            \textbf{Approvazione} & Massarenti Alessandro\\
            \hline
            \textbf{Redazione}& Peron Samuel\\
            & Romano Davide\\
            \hline
            \textbf{Verifica} & Pierobon Luca\\
            & Massarenti Alessandro\\
            & Bonavigo Michele \\
        \end{tabularx}
    \end{center}
}
\newcommand{\mzerounouno}{
    \begin{itemize}
        \item Aggiunto paragrafo relativo all'analisi dei rischi;
    \end{itemize}
}

\newcommand{\vzerozerouno}{
    \hline
    0.1.1 & 10 feb 2023 & \mzerounouno & \pzerounouno\\
}
\newcommand{\pzeroduezero}{
    \begin{center}
        \begin{tabularx}{\linewidth}{l | X}
            \textbf{Approvazione} & Massarenti Alessandro\\
            \hline
            \textbf{Redazione}& Peron Samuel\\
            & Romano Davide\\
            \hline
            \textbf{Verifica} & Pierobon Luca\\
            & Massarenti Alessandro\\
            & Bonavigo Michele \\
        \end{tabularx}
    \end{center}
}
\newcommand{\mzeroduezero}{
    \begin{itemize}
        \item Verifica generale del documento
    \end{itemize}
}

\newcommand{\vzeroduezero}{
    \hline
    0.2.0 & 14 mar 2023 & \mzeroduezero & \pzeroduezero\\
}
\newcommand{\pzerodueuno}{
    \begin{center}
        \begin{tabularx}{\linewidth}{l | X}
            \textbf{Approvazione} & Massarenti Alessandro\\
            \hline
            \textbf{Redazione}& Peron Samuel\\
            & Romano Davide\\
            \hline
            \textbf{Verifica} & Pierobon Luca\\
            & Massarenti Alessandro\\
            & Bonavigo Michele \\
        \end{tabularx}
    \end{center}
}
\newcommand{\mzerodueuno}{
    \begin{itemize}
        \item Aggiunto paragrafo relativo alla pianificazione;
        \item Aggiunti sottoparagrafi relativi ai preventivi per ogni fase pianificata;
    \end{itemize}
}

\newcommand{\vzerodueuno}{
    \hline
    0.2.1 & 8 mar 2023 & \mzerodueuno & \pzerodueuno\\
}
\newcommand{\pzeroduedue}{
    \begin{center}
        \begin{tabularx}{\linewidth}{l | X}
            \textbf{Approvazione} & Massarenti Alessandro\\
            \hline
            \textbf{Redazione}& Peron Samuel\\
            & Romano Davide\\
            \hline
            \textbf{Verifica} & Pierobon Luca\\
            & Massarenti Alessandro\\
            & Bonavigo Michele \\
        \end{tabularx}
    \end{center}
}
\newcommand{\mzeroduedue}{
    \begin{itemize}
        \item Aggiunti consuntivi relativamente a fase di analisi, technology baseline e PoC
    \end{itemize}
}

\newcommand{\vzeroduedue}{
    \hline
    0.2.2 & 13 mar 2023 & \mzeroduedue & \pzeroduedue\\
}
\newcommand{\pzerotrezero}{
    \begin{center}
        \begin{tabularx}{\linewidth}{l | X}
            \textbf{Verifica} & Pierobon Luca\\
            & Massarenti Alessandro\\
            & Bonavigo Michele \\
        \end{tabularx}
    \end{center}
}
\newcommand{\mzerotrezero}{
    \begin{itemize}
        \item Revisione del documento
    \end{itemize}
}

\newcommand{\vzerotrezero}{
    \hline
    0.3.0 & 15 mar 2023 & \mzerotrezero & \pzerotrezero\\
}
\newcommand{\pzerotreuno}{
    \begin{center}
        \begin{tabularx}{\linewidth}{l | X}
            \textbf{Approvazione} & Massarenti Alessandro\\
            \hline
            \textbf{Redazione}& Peron Samuel\\
            & Romano Davide\\
            \hline
            \textbf{Verifica} & Pierobon Luca\\
            & Massarenti Alessandro\\
            & Casarotto Mattia Michele \\
        \end{tabularx}
    \end{center}
}
\newcommand{\mzeroduedue}{
    \begin{itemize}
        \item Sistemate le sezioni relative alla pianificazione ed ai consuntivi di periodo.
    \end{itemize}
}

\newcommand{\vzerotreuno}{
    \hline
    0.3.1 & 16 mar 2023 & \mzerotreuno & \pzerotreuno\\
}
\newcommand{\pzeroquattrozero}{
    \begin{center}
        \begin{tabularx}{\linewidth}{l | X}
            \textbf{Verifica} & Casarotto Mattia\\
        \end{tabularx}
    \end{center}
}
\newcommand{\mzeroquattrozero}{
    \begin{itemize}
        \item Revisione del documento
    \end{itemize}
}

\newcommand{\vzeroquattrozero}{
    \hline
    0.4.0 & DATA & \mzeroquattrozero & \pzeroquattrozero\\
}
\newcommand{\punozerozero}{
    \begin{center}
        \begin{tabularx}{\linewidth}{l | X}
            \textbf{Verifica} & Pierobon Luca\\
            & Massarenti Alessandro\\
            & Bonavigo Michele \\
            & Casarotto Mattia \\
        \end{tabularx}
    \end{center}
}
\newcommand{\mzerotrezero}{
    \begin{itemize}
        \item Revisione del documento
    \end{itemize}
}

\newcommand{\vunozerozero}{
    \hline
    1.0.0 & 23 mar 2023 & \munozerozero & \punozerozero\\
}
\newcommand{\punozerouno}{
    \begin{center}
        \begin{tabularx}{\linewidth}{l | X}
            \textbf{Approvazione} & Pierobon Luca\\
            \hline
            \textbf{Redazione} & Bonavigo Michele\\
            \hline
            \textbf{Verifica} & Casarotto Mattia\\
        \end{tabularx}
    \end{center}
}
\newcommand{\munozerouno}{
    \begin{itemize}
        \item Corretta versione del documento
    \end{itemize}
}

\newcommand{\vunozerouno}{
    \hline
    1.0.1 & 2 apr 2023 & \munozerouno & \punozerouno\\
}

%Tabella
\begin{center}
    \begin{xltabular}{\linewidth}{|l|l|X|X|}
        \hline
        \textbf{Versione} & \textbf{Data} & \textbf{Modifica}& \textbf{Persone}\\
        \vunozerouno
        \vunozerozero
        \vzeroquattrozero
        \vzerotreuno
        \vzerotrezero
        \vzeroduedue
        \vzerodueuno
        \vzeroduezero
        \vzerounouno
        \vzerounozero
        \vzerozerodue
        \vzerozerouno
        \hline
    \end{xltabular}
\end{center}


\tableofcontents
\clearpage
\pagenumbering{arabic}
\chapter{Introduzione}

\section{Scopo del Documento}
Nel seguente documento viene illustrato un prospetto di pianificazione in modo dettagliato e delle modalità attraverso le quali avverrà lo sviluppo del progetto. Il documento tratterà, in ordine, i seguenti punti:
\begin{itemize}
    \item analisi dei rischi;
    \item descrizione del modello di sviluppo adottato;
    \item suddivisione delle varie fasi con conseguente assegnazione dei ruoli;
    \item stima dei costi e delle risorse necessarie.
\end{itemize}

\section{Scopo del prodotto}
L'obiettivo di SWEasabi e dell'azienda Imola Informatica è lo sviluppo di un sistema per l'ottimizzazione dell'illuminazione pubblica che permetta ai gestori di sfruttare la possibilità di regolare l'intensità della luce emessa dagli impianti d'illuminazione. Un sistema così congegnato consentirebbe, da un lato, di garantire sicurezza stradale e sociale, e dall'altro permetterebbe di risparmiare energia, dunque, risorse economiche e ambientali.


\section{Glossario}
Per evitare ambiguità relative alle terminologie utilizzate è stato creato un documento denominato “Glos-
sario”. Questo documento comprende tutti i termini tecnici scelti dai membri del gruppo e utilizzati nei
vari documenti con le relative definizioni. Tutti i termini inclusi in questo glossario, vengono segnalati
all'interno del documento con l'apice^{G} accanto alla parola.

\section{Riferimenti}
\subsection{Riferimenti Normativi}
\begin{itemize}
    \item \textit{Norme-di-progetto};
    \item \href{https://www.math.unipd.it/~tullio/IS-1/2022/Progetto/C2.pdf}{capitolato^{G} d'appalto C2}.
\end{itemize}

\subsection{Riferimenti Informativi}
\begin{itemize}
    \item \textit{Piano-di-Qualifica};
    \item \href{https://www.math.unipd.it/~tullio/IS-1/2022/Dispense/T03.pdf}{Il ciclo di vita del software} - Materiale didattico del corso di Ingegneria del Software:
    \begin{itemize}
        \item modello incrementale - Slides: 19, 20, 21 e 22.
\end{itemize}
\chapter{Analisi dei rischi}
Nel corso dello sviluppo del progetto è naturale incontrare vari tipi di problematiche, che con un’attenta e continua analisi dei rischi possono essere mitigate. Il piano per la gestione dei rischi viene suddiviso in 4 attività:
\begin{itemize}
    \item individuazione dei possibili eventi che possono portare a dei problemi durante l’avanzamento;
    \item analisi del problema, in particolare la probabilità con cui si possa verificare e le conseguenze
    negative che comporta;
    \item pianificazione di misure da prendere per impedire il verificarsi dei rischi e comportamenti da seguire nel caso in cui essi dovessero presentarsi. In questo modo si evita che un rischio possa diventare insostenibile;
    \item monitoraggio continuo dei rischi, cercando di prevenirli o minimizzando l’effetto negativo di quest’ultimi.
\end{itemize}

\section{Rischi tecnologici}

\section{Rischi personali}

\section{Rischi organizzativi}

\section{Rischi legati ai requisiti}

\chapter{Modello}
Come già accennato precedentemente, si è deciso di adottare come modello di sviluppo il modello \textbf{incrementale}
\section{Modello incrementale}
Il modello incrementale prevede rilasci multipli e successivi, quindi vogliamo che ci sia un incremento delle funzionalità dopo ogni rilascio. In questo modo viene ridotto il rischio di fallimento ed il lavoro procederà solo dopo l’accettazione da parte del proponente^{G}. L’instabilità dei requisiti può essere gestita solo tra un rilascio e l’altro, ma comunque con l’approvazione da parte di Imola informatica (i requisti più "importanti" verranno stabilizzati per primi). I principali vantaggi di questo modello sono:
\begin{itemize}
    \item possibilità di presentare al proponente un prodotto sempre funzionante;
    \item si combina bene con il versionamento^{G}, rendendo più visibili le modifiche;
    \item gestione delle priorità tra i vari requisiti, dando priorità a funzionalità primarie;
    \item gli errori sono limitati all’incremento corrente e la loro correzione è più economica;
    \item gli incrementi terminano solo quando verrà accettato il prodotto con quanto di nuovo introdotto, riducendo così la possibilità di trascinare errori durante lo sviluppo del progetto.
\end{itemize}
\chapter{Pianificazione}
La pianificazione è il processo di riflessione sulle attività necessarie per raggiungere un obiettivo desiderato. È la prima e più importante attività per ottenere i risultati desiderati. Implica la creazione e il mantenimento di un piano, come gli aspetti psicologici che richiedono abilità concettuali, come l'uso della logica e dell'immaginazione per visualizzare non solo un risultato finale desiderato, ma i passaggi necessari per ottenere quel risultato. In questo progetto si è scelto di suddividere la pianificazione in 6 fasi: 
\begin{itemize}
    \item Fase di analisi preliminare;
    \item Progettazione Technology Baseline;
    \item Proof of Concept;
    \item Progettazione e codifica requisiti obbligatori;
    \item Progettazione e codifica requisiti opzionali;
    \item Validazione e Test.
\end{itemize}
Essendo il ciclo di vita di un SW rappresentabile come un automa a stati finiti, gli stati saranno le fasi sopra elencate, mentre gli archi saranno rappresentati da tutte le attività che verranno svolte per passare da uno stato all'altro. Di ogni fase andremo inoltre a specificare quando iniziare (pre-condizioni) e quando finire (post-condizioni).
Inoltre, per ogni fase verranno pianificati un preventivo costi e la suddivisione dei ruoli.

\textbf{Suddivisione ruoli:}
Durante la realizzazione di ogni documento, i ruoli saranno assegnati in base alla disponibilità di tempo di ogni membro del gruppo. Ciò significa che un membro del gruppo può assumere lo stesso ruolo per documenti diversi, ma sarebbe responsabile soltanto di un documento alla volta. I responsabili verranno cambiati all'inizio di ogni baseline. Questa stessa suddivisione verrà applicata durante la stesura dell'architettura e l'implementazione del software.

Al termine del progetto didattico, ogni membro del gruppo avrà assunto almeno una volta tutti i ruoli.

\section{Fase di analisi preliminare}

\textbf{Periodo di svolgimento}
Data di inizio: 2022 - 11 - 07 
Data di fine: 2023 - 0x - xx

\textbf{Pre-condizioni:}
\begin{itemize}
    \item Formazione gruppo;
    \item presentazione capitolato.
\end{itemize}

\textbf{Post-condizioni:}
    \begin{itemize}
        \item Determinazione di un way of working interno al gruppo;
        \item Redazione dei documenti:
        \begin{itemize}
            \item analisi dei requisiti;
            \item norme di progetto;
            \item piano di progetto;
            \item piano di qualifica;
            \item glossario.
        \end{itemize}
        \item Verifica dei documenti redatti;
    \end{itemize}

\subsection{Ruoli attivi}
\begin{itemize}
    \item Responsabile
    \item Amministratore
    \item Analista
    \item Verificatore
\end{itemize}

\subsection{Preventivo costi}

\section{Progettazione Technology Baseline}

\section{Proof of Concept};

\section{Progettazione e codifica requisiti obbligatori}

\section{Progettazione e codifica requisiti opzionali}

\section{Validazione e Test}
\chapter{Consuntivo di periodo}
Questa sezione presenta i costi e gli orari effettivi sostenuti dal gruppo SWEasabi ponendo a confronto i costi e gli orari preventivati con quelli effettivi per ogni periodo con lo scopo di monitorare l'andamento del progetto. Verrà presentato, per ogni fase, un bilancio dei costi sostenuti utile alla programmazione delle successive fasi del progetto.

\section{Technology Baseline}

La seguente tabella mostra le ore che ogni persona ha ricoperto per ciascun ruolo:

\begin{table}[ht]
    \begin{tabularx}{\linewidth}{X|rrrrrrr}
    \rowcolor{gray!30}& Re & Amm & An & Pro & Prog & Ver & tot \\
    \hline
    Bonavigo Michele                        & 2 (+2)     & 1 (+1)   & 4         & 6 (+1)    & 0     & 2         & 15 (+4)\\ 

    \rowcolor{gray!10}Casarotto Mattia      & 0          & 3 (+1)   & 4         & 5 (+2)    & 0     & 1         & 13 (+3)\\ 

    Massarenti Alessandro                   & 5 (+1)     & 0        & 4 (+2)    & 4         & 0     & 3 (+3)    & 16 (+6) \\ 

    \rowcolor{gray!10}Peron Samuel          & 0          & 2        & 3         & 2 (-1)    & 0     & 0 (-1)    & 7 (-2) \\  

    Pierobon Luca                           & 4          & 1 (+1)   & 1         & 3         & 0     & 3         & 12 (+1) \\ 

    \rowcolor{gray!10}Romano Davide         & 1 (-2)     & 0        & 1         & 2         & 0     & 3         & 7 (-2) \\ 

    Zarantonello Giorgio                    & 3          & 0        & 2         & 5         & 0     & 0         & 10     \\ 

    \hline                                  & 15 (+1)    & 7 (+3)   & 19 (+2)   & 27 (+2)   & 0     & 12 (+2)   & 80 (+10) \\ 
    \end{tabularx}
\end{table}


La seguente tabella mostra il costo per ciascun ruolo:
\begin{table}[ht]
    \begin{tabularx}{\linewidth}{X|rrrrrrr}
    \rowcolor{gray!30}Ruolo & Ore & Costo \\
    \hline
    Responsabile                            & 15   & 450€ (+30€)\\
    \rowcolor{gray!10}Amministratore        & 7    & 140€ (+60€)\\
    Analista                                & 19   & 475€ (+50€)\\
    \rowcolor{gray!10}Progettista           & 27   & 675€ (+50€)\\
    Programmatore                           & 0    & 0€ \\
    \rowcolor{gray!10}Verificatore          & 12   & 180€ (+30€) \\
    \hline Totale                           & 75   & 1920€ (+220€) \\ 
    \end{tabularx}
\end{table}

\subsection{Motivazione delle variazioni}
Sono state utilizzate più ore rispetto a quelle preventivate, in quanto i problemi riscontrati nella fase precedente hanno portato ad un avanzo di ore che sono state in parte riutilizzate in questa fase.

\subsection{Bilancio finale}
Il bilancio finale è stato di 1920€, il costo totale è aumentato di 220€ rispetto al bilancio iniziale. Gli obiettivi sono stati raggiunti ma sono state utilizzate più risorse (tempo-denaro) rispetto a quanto preventivato, il bilancio finale è negativo ma non si rende necessaria una ripianificazione dei prossimi periodi.



\end{document}
