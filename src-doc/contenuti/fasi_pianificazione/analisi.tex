\section{Analisi}

\textbf{Periodo di svolgimento}
\\ Da incremento 1 a incremento 4

\textbf{Attività da svolgere}
    \begin{itemize}
        \item \textbf{Miglioramento del way of working:} approccio ai metodi e alle procedure che una persona o un team utilizza per svolgere il proprio lavoro. Comprende le varie abitudini, routine e pratiche che guidano come viene svolto il lavoro, compresa la comunicazione, la collaborazione, la presa di decisioni e la risoluzione dei problemi;
        \item \textbf{analisi dei requisiti:} processo di identificazione, raccolta, documentazione, e verifica dei requisiti di un sistema, prodotto o servizio. Questa attività si concentra sul comprendere le esigenze degli utenti, degli stakeholder e del mercato per determinare le funzionalità, le prestazioni e le caratteristiche necessarie del sistema o del prodotto;
        \item \textbf{norme di progetto:} insieme di regole, procedure e linee guida che definiscono i processi di sviluppo, le metodologie di lavoro e le responsabilità degli stakeholder all'interno di un progetto. Queste norme sono stabilite per garantire la coerenza, la qualità e la conformità del lavoro svolto durante il progetto;
        \item \textbf{piano di progetto:} documento che definisce le attività, le risorse, le tempistiche, i costi e le responsabilità necessarie per gestire e completare con successo un progetto. Il piano di progetto fornisce una roadmap di come il progetto verrà pianificato, eseguito e controllato, nonché come verranno gestiti i rischi e le eventuali variazioni rispetto al piano originale;
        \item \textbf{piano di qualifica:} documento che descrive gli obiettivi, le strategie e le tecniche utilizzate per garantire che il prodotto o il servizio finale soddisfi i requisiti di qualità concordati. Il piano di qualifica fornisce una panoramica dei processi di verifica e validazione del prodotto, inclusi i criteri di accettazione, i test funzionali e non funzionali, le procedure di verifica della conformità e le metriche di qualità;
        \item \textbf{glossario:} documento che fornisce una definizione concisa e chiara di ciascun termine, aiutando a comprendere il significato di parole che potrebbero essere poco familiari o ambigue.
    \end{itemize}

\subsection{Ruoli attivi}
\begin{itemize}
    \item Responsabile 
    \item Amministratore 
    \item Analista 
    \item Progettista
    \item Verificatore
\end{itemize}

\subsection{Preventivo costi}

La seguente tabella mostra la suddivisione delle ore che ogni persona ricoprirà per ciascun ruolo:

\begin{table}[H]
    \begin{tabularx}{\linewidth}{X|rrrrrrr}
    \rowcolor{gray!30}& Re & Amm & An & Pro & Prog & Ver & tot \\
    \hline
    Bonavigo Michele                        & 5 & 0 & 8 & 2 & 0 & 6 & 21 \\
    \rowcolor{gray!10}Casarotto Mattia      & 5 & 0 & 9 & 2 & 0 & 5 & 21 \\
    Massarenti Alessandro                   & 0 & 6 & 9 & 3 & 0 & 6 & 24 \\
    \rowcolor{gray!10}Peron Samuel          & 6 & 0 & 9 & 1 & 0 & 5 & 21 \\
    Pierobon Luca                           & 0 & 6 & 10 & 2 & 0 & 4 & 22 \\
    \rowcolor{gray!10}Romano Davide         & 0 & 5 & 9 & 2 & 0 & 4 & 20 \\
    Zarantonello Giorgio                    & 0 & 4 & 11 & 2 & 0 & 5 & 22 \\
    \hline                                  & 16 & 21 & 65 & 14 & 0 & 35 & 151 \\ 
    \end{tabularx}
\end{table}

La seguente tabella mostra il costo per ciascun ruolo:
\begin{table}[H]
    \begin{tabularx}{\linewidth}{X|rrrrrrr}
    \rowcolor{gray!30}Ruolo & Ore & Costo \\
    \hline
    Responsabile                            & 16 & 480€ \\
    \rowcolor{gray!10}Amministratore        & 21 & 420€ \\
    Analista                                & 65 & 1625€ \\
    \rowcolor{gray!10}Progettista           & 14 & 350€ \\
    Programmatore                           & 0 & 0€ \\
    \rowcolor{gray!10}Verificatore          & 35 & 525€ \\
    \hline Totale                           & 151 & 3400€ \\ 
    \end{tabularx}
\end{table}

Grafico della distribuzione percentuale delle ore per ruolo:
\begin{center}
\begin{tikzpicture}
    \pie[color = {
        yellow!90!black,
        green!60!black,
        blue!60, red!70,
        cyan!70,}
    ]{
        10.59/Responsabile,
        13.90/Amministratore,
        43.04/Analista,
        9.27/Progettista,
        23.17/Verificatore
    }
     
    \end{tikzpicture}
\end{center}