\section{Progettazione e codifica della soluzione ai requisiti obbligatori}

\textbf{Periodo di svolgimento}
\\ Da incremento 6 a incremento 10

\textbf{Attività da svolgere}
    \begin{itemize}
        \item \textbf{Diagrammi di classe:} rappresentazione grafica del modello di classi e delle relazioni tra di esse in un software, mostrano le classi, i loro attributi e i loro metodi, così come le relazioni tra di esse, come l'ereditarietà, l'associazione, la composizione e l'aggregazione. Inoltre, i diagrammi di classe possono mostrare le interfacce delle classi e le relazioni tra le interfacce; 
        \item \textbf{diagrammi di attività:} descrivono il flusso di lavoro di un processo o di una procedura all'interno del software e mostrano un insieme di azioni, decisioni e attività che sono necessarie per raggiungere uno specifico obiettivo all'interno del sistema;
        \item \textbf{sviluppo codice:} sviluppo del codice per la realizzazione del software richiesto provvisto dei requisiti obbligatori imposti dal committente;
        \item \textbf{test:} creazione e utilizzo di test per verificare il corretto funzionamento del codice sviluppato;
        \item \textbf{manuale utente:} manuale che descrive l'utilizzo del software utile per l'utente finale.
    \end{itemize}

\subsection{Ruoli attivi}
\begin{itemize}
    \item Responsabile 
    \item Amministratore
    \item Analista
    \item Progettista 
    \item Programmatore 
    \item Verificatore 
\end{itemize}

\subsection{Preventivo costi}

La seguente tabella mostra la suddivisione delle ore che ogni persona ricoprirà per ciascun ruolo:

\begin{table}[H]
    \begin{tabularx}{\linewidth}{X|rrrrrrr}
    \rowcolor{gray!30}& Re & Amm & An & Pro & Prog & Ver & tot \\
    \hline
    Bonavigo Michele                        & 0 & 3 & 0 & 9 & 12 & 6 & 30 \\
    \rowcolor{gray!10}Casarotto Mattia      & 4 & 2 & 0 & 9 & 10 & 5 & 30 \\
    Massarenti Alessandro                   & 5 & 2 & 0 & 8 & 10 & 6 & 31 \\
    \rowcolor{gray!10}Peron Samuel          & 0 & 3 & 5 & 8 & 11 & 4 & 31 \\
    Pierobon Luca                           & 0 & 2 & 5 & 9 & 10 & 3 & 29 \\
    \rowcolor{gray!10}Romano Davide         & 3 & 4 & 5 & 7 & 10 & 1 & 30 \\
    Zarantonello Giorgio                    & 3 & 4 & 0 & 5 & 11 & 5 & 28 \\
    \hline                                  & 15 & 20 & 15 & 55 & 74 & 30 & 209 \\ 
    \end{tabularx}
\end{table}

La seguente tabella mostra il costo per ciascun ruolo:
\begin{table}[H]
    \begin{tabularx}{\linewidth}{X|rrrrrrr}
    \rowcolor{gray!30}Ruolo & Ore & Costo \\
    \hline
    Responsabile                            & 15  & 450€ \\
    \rowcolor{gray!10}Amministratore        & 20 & 400€ \\
    Analista                                & 15  & 375€ \\
    \rowcolor{gray!10}Progettista           & 55  & 1375€ \\
    Programmatore                           & 74 & 1110€ \\
    \rowcolor{gray!10}Verificatore          & 30  & 450€ \\
    \hline Totale                           & 209  & 4160€ \\ 
    \end{tabularx}
\end{table}

Grafico della distribuzione percentuale delle ore per ruolo:
\begin{center}
\begin{tikzpicture}
    \pie[color = {
        yellow!90!black,
        green!60!black,
        blue!60, red!70,
        cyan!60, magenta!60}
    ]{
        7.17/Responsabile,
        9.56/Amministratore,
        7.17/Analista,
        26.31/Progettista,
        35.40/Programmatore,
        14.35/Verificatore
    }
     
    \end{tikzpicture}
\end{center}

