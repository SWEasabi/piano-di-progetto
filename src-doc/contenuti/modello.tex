\chapter{Modello}
Come già accennato precedentemente, si è deciso di adottare come modello di sviluppo il modello \textbf{incrementale}
\section{Modello incrementale}
Il modello incrementale prevede rilasci multipli e successivi, quindi vogliamo che ci sia un incremento delle funzionalità dopo ogni rilascio. In questo modo viene ridotto il rischio di fallimento ed il lavoro procederà solo dopo l’accettazione da parte del proponente. L’instabilità dei requisiti può essere gestita solo tra un rilascio e l’altro, ma comunque con l’approvazione da parte di Imola informatica (i requisti più "importanti" verranno stabilizzati per primi). I principali vantaggi di questo modello sono:
\begin{itemize}
    \item possibilità di presentare al proponente un prodotto sempre funzionante;
    \item si combina bene con il versionamento, rendendo più visibili le modifiche;
    \item gestione delle priorità tra i vari requisiti, dando priorità a funzionalità primarie;
    \item gli errori sono limitati all’incremento corrente e la loro correzione è più economica;
    \item gli incrementi terminano solo quando verrà accettato il prodotto con quanto di nuovo introdotto, riducendo così la possibilità di trascinare errori durante lo sviluppo del progetto.
\end{itemize}