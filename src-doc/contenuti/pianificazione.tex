\chapter{Pianificazione}
La pianificazione è il processo di riflessione sulle attività necessarie per raggiungere un obiettivo desiderato. È la prima e più importante attività per ottenere i risultati desiderati. Implica la creazione e il mantenimento di un piano, come gli aspetti psicologici che richiedono abilità concettuali, come l'uso della logica e dell'immaginazione per visualizzare non solo un risultato finale desiderato, ma i passaggi necessari per ottenere quel risultato. In questo progetto si è scelto di suddividere la pianificazione in 6 fasi: 
\begin{itemize}
    \item Analisi;
    \item Technology Baseline;
    \item Proof of Concept;
    \item Progettazione e codifica della soluzione ai requisiti obbligatori;
    \item Progettazione e codifica della soluzione ai requisiti opzionali;
    \item Verifica e collaudo.
\end{itemize}
In ogni fase vengono specificate le attività da svolgere, i ruoli attivi e il costo preventivato. La durata di ogni fase è indicata dal numero di incrementi, un incremento è un periodo di tempo di 2 settimane.

\textbf{Suddivisione ruoli:}
Durante la realizzazione di ogni documento, i ruoli saranno assegnati in base alla disponibilità di tempo di ogni membro del gruppo. Ciò significa che un membro del gruppo può assumere lo stesso ruolo per documenti diversi, ma sarebbe responsabile soltanto di un documento alla volta. I responsabili verranno cambiati all'inizio di ogni baseline. Questa stessa suddivisione verrà applicata durante la stesura dell'architettura e l'implementazione del software.

Al termine del progetto didattico, ogni membro del gruppo avrà assunto almeno una volta tutti i ruoli, raggiungendo il numero di ore previste nella tabella sottostante:

\begin{table}[H]
    \begin{tabularx}{\linewidth}{X|rrrrrrr}
    \rowcolor{gray!30}& Re & Amm & An & Pro & Prog & Ver & tot \\
    \hline
    Bonavigo Michele                        & 9 & 13 & 18 & 21 & 18 & 16 & 95 \\
    \rowcolor{gray!10}Casarotto Mattia      & 10 & 12 & 18 & 21 & 18 & 16 & 95 \\
    Massarenti Alessandro                   & 10 & 10 & 18 & 22 & 19 & 16 & 95 \\
    \rowcolor{gray!10}Peron Samuel          & 9 & 12 & 19 & 20 & 18 & 17 & 95 \\
    Pierobon Luca                           & 10 & 11 & 18 & 20 & 17 & 19 & 95 \\
    \rowcolor{gray!10}Romano Davide         & 10 & 12 & 17 & 21 & 18 & 17 & 95 \\
    Zarantonello Giorgio                    & 9 & 10 & 19 & 22 & 17 & 18 & 95 \\
    \hline                                  & 67 & 80 & 127 & 147 & 125 & 119 & 665 \\ 
    \end{tabularx}
\end{table}

Ad ogni ruolo corrisponde la seguente tariffa oraria:

\begin{table}[H]
    \begin{tabularx}{\linewidth}{X|rrrrrrr}
    \rowcolor{gray!30}Ruolo & Costo \\
    \hline
    Responsabile                       & 30€ \\
    \rowcolor{gray!10}Amministratore   & 20€ \\
    Analista                           & 25€ \\
    \rowcolor{gray!10}Progettista       & 25€ \\
    Programmatore                       & 15€ \\
    \rowcolor{gray!10}Verificatore      & 15€ \\
    \hline
    \end{tabularx}
\end{table}

I costi di produzione si attestano a 14090€ con data di consegna entro e non oltre il 28 Aprile 2023.

\section{Analisi}

La seguente tabella mostra le ore che ogni persona ha ricoperto per ciascun ruolo:
\begin{table}[ht]
    \begin{tabularx}{\linewidth}{X|rrrrrrr}
    \rowcolor{gray!30}& Re & Amm & An & Pro & Prog & Ver & tot \\
    \hline
    Bonavigo Michele                        & 4 (-1)   & 0          & 7 (-1)       & 1 (-1)     & 0     & 5 (-1)   & 18 (-3) \\

    \rowcolor{gray!10}Casarotto Mattia      & 7 (+2)   & 0          & 6 (-3)       & 2          & 0     & 2 (-3)   & 17 (-4) \\

    Massarenti Alessandro                   & 0        & 6          & 9            & 3          & 0     & 6        & 24 \\ 

    \rowcolor{gray!10}Peron Samuel          & 6        & 0          & 7 (-2)       & 1          & 0     & 4 (-1)   & 18 (-3) \\ 

    Pierobon Luca                           & 0        & 7 (+1)     & 9 (-1)       & 2          & 0     & 4        & 22 \\ 

    \rowcolor{gray!10}Romano Davide         & 0        & 5          & 6 (-3)       & 0 (-2)     & 0     & 2 (-2)   & 13 (-7)\\

    Zarantonello Giorgio                    & 0        & 2 (-2)     & 7 (-4)       & 1 (-1)     & 0     & 2 (-3)   & 12 (-10)\\

    \hline                                  & 17 (+1)  & 20 (-1)    & 51 (-14)     & 11 (-3)    & 0     & 25 (-10)  & 124 (-27)\\  
    \end{tabularx}
\end{table}

La seguente tabella mostra il costo per ciascun ruolo:
\begin{table}[ht]
    \begin{tabularx}{\linewidth}{X|rrrrrrr}
    \rowcolor{gray!30}Ruolo & Ore & Costo \\
    \hline
    Responsabile                            & 17 & 510€ (+30€)\\
    \rowcolor{gray!10}Amministratore        & 20 & 400€ (-20€) \\
    Analista                                & 51 & 1275€ (-350€)\\
    \rowcolor{gray!10}Progettista           & 11 & 275€ (-75€) \\
    Programmatore                           & 0 & 0€ \\
    \rowcolor{gray!10}Verificatore          & 25 & 375€ (-150€) \\
    \hline Totale                           & 124 & 2835€ (-565€) \\ 
    \end{tabularx}
\end{table}

\subsection{Motivazione delle variazioni}

Alcune attività si sono rivelate essere più lunghe dei singoli sprint quindi ci sono state difficoltà nell'organizzare il lavoro in modo da poterlo suddividere in attività più piccole, in primis per la difficoltà nel conciliare gli impegni personali e successivamente nella gestione delle attività durante il periodo delle vacanze invernali che anticipavano la sessione d'esame.  
Un altro problema significativo è stato il cambio del supporto hardware da parte del proponente per problemi di reperimento dello stesso, questo ha comportato una modifica nell'analisi dei requisiti.

\subsection{Bilancio finale}

Il bilancio finale è stato di 2835€, il costo totale è stato ridotto di 565€ rispetto al bilancio iniziale. Anche se gli obiettivi sono stati raggiunti con un leggero ritardo per le cause sopra descritte, possiamo ritenere il bilancio finale comunque positivo non rendendo quindi necessaria una ripianificazione dei prossimi periodi, richiedendo però una riorganizzazione dei singoli sprint.

\section{Technology Baseline}

La seguente tabella mostra le ore che ogni persona ha ricoperto per ciascun ruolo:

\begin{table}[ht]
    \begin{tabularx}{\linewidth}{X|rrrrrrr}
    \rowcolor{gray!30}& Re & Amm & An & Pro & Prog & Ver & tot \\
    \hline
    Bonavigo Michele                        & 2 (+2)     & 1 (+1)   & 4         & 6 (+1)    & 0     & 2         & 15 (+4)\\ 

    \rowcolor{gray!10}Casarotto Mattia      & 0          & 3 (+1)   & 4         & 5 (+2)    & 0     & 1         & 13 (+3)\\ 

    Massarenti Alessandro                   & 5 (+1)     & 0        & 4 (+2)    & 4         & 0     & 3 (+3)    & 16 (+6) \\ 

    \rowcolor{gray!10}Peron Samuel          & 0          & 2        & 3         & 2 (-1)    & 0     & 0 (-1)    & 7 (-2) \\  

    Pierobon Luca                           & 4          & 1 (+1)   & 1         & 3         & 0     & 3         & 12 (+1) \\ 

    \rowcolor{gray!10}Romano Davide         & 1 (-2)     & 0        & 1         & 2         & 0     & 3         & 7 (-2) \\ 

    Zarantonello Giorgio                    & 3          & 0        & 2         & 5         & 0     & 0         & 10     \\ 

    \hline                                  & 15 (+1)    & 7 (+3)   & 19 (+2)   & 27 (+2)   & 0     & 12 (+2)   & 80 (+10) \\ 
    \end{tabularx}
\end{table}


La seguente tabella mostra il costo per ciascun ruolo:
\begin{table}[ht]
    \begin{tabularx}{\linewidth}{X|rrrrrrr}
    \rowcolor{gray!30}Ruolo & Ore & Costo \\
    \hline
    Responsabile                            & 15   & 450€ (+30€)\\
    \rowcolor{gray!10}Amministratore        & 7    & 140€ (+60€)\\
    Analista                                & 19   & 475€ (+50€)\\
    \rowcolor{gray!10}Progettista           & 27   & 675€ (+50€)\\
    Programmatore                           & 0    & 0€ \\
    \rowcolor{gray!10}Verificatore          & 12   & 180€ (+30€) \\
    \hline Totale                           & 75   & 1920€ (+220€) \\ 
    \end{tabularx}
\end{table}

\subsection{Motivazione delle variazioni}
Sono state utilizzate più ore rispetto a quelle preventivate, in quanto i problemi riscontrati nella fase precedente hanno portato ad un avanzo di ore che sono state in parte riutilizzate in questa fase.

\subsection{Bilancio finale}
Il bilancio finale è stato di 1920€, il costo totale è aumentato di 220€ rispetto al bilancio iniziale. Gli obiettivi sono stati raggiunti ma sono state utilizzate più risorse (tempo-denaro) rispetto a quanto preventivato, il bilancio finale è negativo ma non si rende necessaria una ripianificazione dei prossimi periodi.


\section{Proof of Concept}

La seguente tabella mostra le ore che ogni persona ha ricoperto per ciascun ruolo:
\begin{table}[H]
    \begin{tabularx}{\linewidth}{X|rrrrrrr}
    \rowcolor{gray!30}& Re & Amm & An & Pro & Prog & Ver & tot \\
    \hline
    Bonavigo Michele                        & 4 (+2)     & 6        & 4 (+2)    & 0         & 0        & 0         & 14 (+4) \\

    \rowcolor{gray!10}Casarotto Mattia      & 0          & 5        & 3         & 0         & 2        & 0         & 10 \\

    Massarenti Alessandro                   & 1 (+1)     & 2 (+2)   & 5         & 4         & 2 (+2)   & 0         & 14 (+5) \\

    \rowcolor{gray!10}Peron Samuel          & 0          & 4        & 2         & 0         & 0        & 3         & 9 \\ 

    Pierobon Luca                           & 3          & 0        & 0         & 2 (+2)    & 3 (+1)   & 5 (+1)    & 13 (+4) \\ 

    \rowcolor{gray!10}Romano Davide         & 2 (-1)     & 0        & 0         & 2         & 0        & 3         & 7 (-1) \\ 

    Zarantonello Giorgio                    & 0          & 0        & 3         & 3 (-1)    & 2        & 0         & 8 (-1) \\ 

    \hline                                  & 10 (+2)    & 17 (+2)  & 17 (+2)   & 11 (+1)   & 9 (+3)   & 11 (+1)   & 75 (+11) \\  
    \end{tabularx}
\end{table} 

La seguente tabella mostra il costo per ciascun ruolo:
\begin{table}[H]
    \begin{tabularx}{\linewidth}{X|rrrrrrr}
    \rowcolor{gray!30}Ruolo & Ore & Costo \\
    \hline
    Responsabile                            & 10    & 300€ (+60€)\\
    \rowcolor{gray!10}Amministratore        & 17    & 340€ (+40€)\\
    Analista                                & 17    & 425€ (+50€)\\
    \rowcolor{gray!10}Progettista           & 11    & 275€ (+25€) \\
    Programmatore                           & 9     & 135€ (+45€) \\
    \rowcolor{gray!10}Verificatore          & 11    & 165€ (+15€)\\
    \hline Totale                           & 75    & 1640€ (+235€) \\ 
    \end{tabularx}
\end{table}

\subsection{Motivazione delle variazioni}

Il proponente ha richiesto che il PoC fosse realizzato su un supporto hardware diverso da quello originariamente previsto, questo ha comportato un ritardo nella realizzazione del PoC di 1 incremento.

\subsection{Bilancio finale}

Il bilancio finale è stato di 1640€, il costo totale è stato aumentato di 235€ rispetto al bilancio iniziale. Anche se gli obiettivi sono stati raggiunti in ritardo per le cause sopra descritte. Il bilancio finale risulta negativo ma non si rende necessaria una ripianificazione dei periodi successivi.

\section{Progettazione e codifica della soluzione ai requisiti obbligatori}

La seguente tabella mostra le ore che ogni persona ha ricoperto per ciascun ruolo:

\begin{table}[ht]
    \begin{tabularx}{\linewidth}{X|rrrrrrr}
    \rowcolor{gray!30}& Re & Amm & An & Pro & Prog & Ver & tot \\
    \hline
    Bonavigo Michele                        & 0      & 3         & 0         & 9      & 12     & 1(-5)    & 25 (-5) \\ 
 
    \rowcolor{gray!10}Casarotto Mattia      & 4      & 2         & 0         & 9      & 10     & 4(-1)    & 29 (-1)\\ 
 
    Massarenti Alessandro                   & 3(-2)  & 0(-2)     & 0         & 8      & 10     & 2(-4)    & 23(-8) \\ 
 
    \rowcolor{gray!10}Peron Samuel          & 0      & 3         & 5         & 8      & 10(-1) & 4        & 30(-1) \\  
 
    Pierobon Luca                           & 0      & 2         & 4(-1)     & 6(-3)  & 9(-1)  & 3        & 24(-5) \\ 
 
    \rowcolor{gray!10}Romano Davide         & 3      & 4         & 5         & 7      & 10     & 2(+1)    & 31(+1) \\ 
 
    Zarantonello Giorgio                    & 3      & 4         & 0         & 5      & 11     & 8(+3)    & 31(+3) \\ 
 
    \hline                                  & 13(-2) & 18(-2)    & 14(-1)   & 52(-3)  & 72(-2) & 24(-6)   & 193(-16) \\ 
    \end{tabularx}
\end{table}

La seguente tabella mostra il costo per ciascun ruolo:
\begin{table}[H]
    \begin{tabularx}{\linewidth}{X|rrrrrrr}
    \rowcolor{gray!30}Ruolo & Ore & Costo \\
    \hline
    Responsabile                            & 13    & 390€ (-60€)\\
    \rowcolor{gray!10}Amministratore        & 18    & 360€ (-40€)\\
    Analista                                & 14    & 350€ (-25€)\\
    \rowcolor{gray!10}Progettista           & 52    & 1300€ (-75€) \\
    Programmatore                           & 72    & 1080€ (-30€) \\
    \rowcolor{gray!10}Verificatore          & 24    & 360€ (-90€)\\
    \hline Totale                           & 193   & 3840€ (-320€) \\ 
    \end{tabularx}
\end{table}

\subsection{Motivazione delle variazioni}

Le ore impiegate sono risultate essere inferiori a quelle preventivate, grazie a un buona progettazione si sono risparmiate ore di implementazione. Durante questa fase si sono riscontrati problemi organizzativi e comunicativi all'interno del gruppo portando a una dilatazione dei tempi di sviluppo e di conseguenza a gravi ritardi.

\subsection{Bilancio finale}

Il bilancio finale è stato di 3840€, il costo totale è stato ridotto di 320€ rispetto al bilancio iniziale. Anche se gli obiettivi sono stati raggiunti con un grave ritardo per le cause sopra descritte, il bilancio finale rimane comunque positivo.

\section{Progettazione e codifica della soluzione ai requisiti opzionali}

\textbf{Periodo di svolgimento}
\\ Da incremento 10 a incremento 13

\textbf{Attività da svolgere}
    \begin{itemize}
        \item \textbf{Progettazione requisiti opzionali:} studio e ricerca di requisiti opzionali e come implementarli nel prodotto finale; 
        \item \textbf{codifica requisiti:} sviluppo di codice per implementare i requisiti opzionali;
        \item \textbf{test:} creazione e utilizzo di test per verificare il corretto funzionamento del codice sviluppato;
        \item \textbf{modifica documenti:} vengono applicate revisioni a documenti esistenti, in cui vengono apportate correzioni, aggiornamenti, aggiunte o rimozioni di contenuti.
    \end{itemize}

\subsection{Ruoli attivi}
\begin{itemize}
    \item Responsabile 
    \item Amministratore 
    \item Analista 
    \item Progettista 
    \item Programmatore 
    \item Verificatore 
\end{itemize}

\subsection{Preventivo costi}

La seguente tabella mostra la suddivisione delle ore che ogni persona ricoprirà per ciascun ruolo:

\begin{table}[H]
    \begin{tabularx}{\linewidth}{X|rrrrrrr}
    \rowcolor{gray!30}& Re & Amm & An & Pro & Prog & Ver & tot \\
    \hline
    Bonavigo Michele                        & 0 & 4 & 4 & 5 & 4 & 0 & 17 \\
    \rowcolor{gray!10}Casarotto Mattia      & 1 & 3 & 2 & 7 & 6 & 0 & 19 \\
    Massarenti Alessandro                   & 0 & 2 & 2 & 3 & 7 & 2 & 16 \\
    \rowcolor{gray!10}Peron Samuel          & 2 & 3 & 0 & 8 & 7 & 0 & 20 \\
    Pierobon Luca                           & 2 & 0 & 2 & 6 & 5 & 4 & 19 \\
    \rowcolor{gray!10}Romano Davide         & 0 & 3 & 2 & 8 & 7 & 3 & 23 \\
    Zarantonello Giorgio                    & 3 & 0 & 3 & 6 & 4 & 5 & 21 \\
    \hline                                  & 8 & 15 & 15 & 43 & 40 & 14 & 135 \\ 
    \end{tabularx}
\end{table}

La seguente tabella mostra il costo per ciascun ruolo:
\begin{table}[H]
    \begin{tabularx}{\linewidth}{X|rrrrrrr}
    \rowcolor{gray!30}Ruolo & Ore & Costo \\
    \hline
    Responsabile                            & 8  & 240€ \\
    \rowcolor{gray!10}Amministratore        & 15  & 300€ \\
    Analista                                & 15  & 375€ \\
    \rowcolor{gray!10}Progettista           & 43  & 1075€ \\
    Programmatore                           & 40  & 600€ \\
    \rowcolor{gray!10}Verificatore          & 14  & 210€ \\
    \hline Totale                           & 135  & 2800€ \\ 
    \end{tabularx}
\end{table}

Grafico della distribuzione percentuale delle ore per ruolo:
\begin{center}
\begin{tikzpicture}
    \pie[color = {
        yellow!90!black,
        green!60!black,
        blue!60, red!70,
        cyan!60, magenta!60}
    ]{
        5.92/Responsabile,
        11.11/Amministratore,
        11.11/Analista,
        31.85/Progettista,
        29.62/Programmatore,
        10.39/Verificatore
    }
     
    \end{tikzpicture}
\end{center}



\section{Verifica e collaudo}

Fase non ancora terminata.
