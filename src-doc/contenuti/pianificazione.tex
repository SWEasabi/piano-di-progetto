\chapter{Pianificazione}
La pianificazione è il processo di riflessione sulle attività necessarie per raggiungere un obiettivo desiderato. È la prima e più importante attività per ottenere i risultati desiderati. Implica la creazione e il mantenimento di un piano, come gli aspetti psicologici che richiedono abilità concettuali, come l'uso della logica e dell'immaginazione per visualizzare non solo un risultato finale desiderato, ma i passaggi necessari per ottenere quel risultato. In questo progetto si è scelto di suddividere la pianificazione in 6 fasi: 
\begin{itemize}
    \item Fase di analisi preliminare;
    \item Progettazione Technology Baseline;
    \item Proof of Concept;
    \item Progettazione e codifica requisiti obbligatori;
    \item Progettazione e codifica requisiti opzionali;
    \item Validazione e Test.
\end{itemize}
In ogni fase vengono specificate le attività da svolgere, i ruoli attivi e il costo preventivato.

\textbf{Suddivisione ruoli:}
Durante la realizzazione di ogni documento, i ruoli saranno assegnati in base alla disponibilità di tempo di ogni membro del gruppo. Ciò significa che un membro del gruppo può assumere lo stesso ruolo per documenti diversi, ma sarebbe responsabile soltanto di un documento alla volta. I responsabili verranno cambiati all'inizio di ogni baseline. Questa stessa suddivisione verrà applicata durante la stesura dell'architettura e l'implementazione del software.

Al termine del progetto didattico, ogni membro del gruppo avrà assunto almeno una volta tutti i ruoli.

\section{Fase di analisi preliminare}

\textbf{Periodo di svolgimento}
\\ Data di inizio: 2022 - 11 - 07 \\ Data di fine: 2023 - 01 - 02

\textbf{Attività da svolgere}
    \begin{itemize}
        \item \textbf{Scelta del way of working :} approccio ai metodi e alle procedure che una persona o un team utilizza per svolgere il proprio lavoro. Comprende le varie abitudini, routine e pratiche che guidano come viene svolto il lavoro, compresa la comunicazione, la collaborazione, la presa di decisioni e la risoluzione dei problemi;
        \item \textbf{analisi dei requisiti :} processo di identificazione, raccolta, documentazione, e verifica dei requisiti di un sistema, prodotto o servizio. Questa attività si concentra sul comprendere le esigenze degli utenti, degli stakeholder e del mercato per determinare le funzionalità, le prestazioni e le caratteristiche necessarie del sistema o del prodotto;
        \item \textbf{norme di progetto :} insieme di regole, procedure e linee guida che definiscono i processi di sviluppo, le metodologie di lavoro e le responsabilità degli stakeholder all'interno di un progetto. Queste norme sono stabilite per garantire la coerenza, la qualità e la conformità del lavoro svolto durante il progetto;
        \item \textbf{piano di progetto:} documento che definisce le attività, le risorse, le tempistiche, i costi e le responsabilità necessarie per gestire e completare con successo un progetto. Il piano di progetto fornisce una roadmap di come il progetto verrà pianificato, eseguito e controllato, nonché come verranno gestiti i rischi e le eventuali variazioni rispetto al piano originale;        \item \textbf{piano di qualifica:} documento che descrive gli obiettivi, le strategie e le tecniche utilizzate per garantire che il prodotto o il servizio finale soddisfi i requisiti di qualità concordati. Il piano di qualifica fornisce una panoramica dei processi di verifica e validazione del prodotto, inclusi i criteri di accettazione, i test funzionali e non funzionali, le procedure di verifica della conformità e le metriche di qualità;
        \item \textbf{glossario:} documento che fornisce una definizione concisa e chiara di ciascun termine, aiutando a comprendere il significato di parole che potrebbero essere poco familiari o ambigue.
    \end{itemize}

\subsection{Ruoli attivi}
\begin{itemize}
    \item Responsabile 
    \item Amministratore 
    \item Analista 
    \item Verificatore
\end{itemize}

\subsection{Preventivo costi}

La seguente tabella mostra la suddivisione delle ore che ogni persona ricoprirà per ciascun ruolo:

\begin{table}[ht]
    \begin{tabularx}{\linewidth}{X|rrrrrrr}
    \rowcolor{gray!30}& Re & Amm & An & Pro & Prog & Ver & tot \\
    \hline
    Bonavigo Michele                        & 5 & 6 & 10 & 0 & 0 & 4 & 25 \\
    \rowcolor{gray!10}Casarotto Mattia      & 8 & 4 & 9 & 0 & 0 & 3 & 23 \\
    Massarenti Alessandro                   & 0 & 8 & 12 & 0 & 0 & 6 & 26 \\
    \rowcolor{gray!10}Peron Samuel          & 5 & 0 & 12 & 0 & 0 & 6 & 23 \\
    Pierobon Luca                           & 0 & 9 & 12 & 0 & 0 & 8 & 29 \\
    \rowcolor{gray!10}Romano Davide         & 0 & 0 & 13 & 0 & 0 & 6 & 19 \\
    Zarantonello Giorgio                    & 0 & 0 & 14 & 0 & 0 & 2 & 16 \\
    \hline                                  & 18 & 27 & 82 & 0 & 0 & 35 & 162 \\ 
    \end{tabularx}
\end{table}

La seguente tabella mostra il costo per ciascun ruolo:
\begin{table}[ht]
    \begin{tabularx}{\linewidth}{X|rrrrrrr}
    \rowcolor{gray!30}Ruolo & Ore & Costo \\
    \hline
    Responsabile                            & 18 & 540€ \\
    \rowcolor{gray!10}Amministratore        & 27 & 540€ \\
    Analista                                & 82 & 2050€ \\
    \rowcolor{gray!10}Progettista           & 0 & 0€ \\
    Programmatore                           & 0 & 0€ \\
    \rowcolor{gray!10}Verificatore          & 35 & 525€ \\
    \hline Totale                           & 162 & 3655€ \\ 
    \end{tabularx}
\end{table}

Grafico della distribuzione percentuale delle ore per ruolo:
\begin{center}
\begin{tikzpicture}
    \pie[
        color = {
            yellow!90!black,
            green!60!black,
            blue!60, red!70}
    ]{
        11.11/Responsabile,
        50.62/Analista,
        16.67/Amministratore,
        21.60/Verificatore
    }
     
    \end{tikzpicture}
\end{center}

\section{Progettazione Technology Baseline}

\textbf{Periodo di svolgimento}
\\ Data di inizio: 2023 - 01 - 03 \\ Data di fine: 2023 - 02 - 05

\textbf{Attività da svolgere}
    \begin{itemize}
        \item \textbf{Studio progettazione tecnologie per Proof of Concept:} Vengono studiate e selezionate le tecnologie necessarie per creare il Proof of Concept; 
        \item \textbf{Modifiche ai documenti:} vengono applicate revisioni a documenti esistenti, in cui vengono apportate correzioni, aggiornamenti, aggiunte o rimozioni di contenuti.
    \end{itemize}

\subsection{Ruoli attivi}
\begin{itemize}
    \item Responsabile
    \item Amministratore 
    \item Analista 
    \item Progettista 
    \item Verificatore 
\end{itemize}

\subsection{Preventivo costi}

La seguente tabella mostra la suddivisione delle ore che ogni persona ricoprirà per ciascun ruolo:

\begin{table}[ht]
    \begin{tabularx}{\linewidth}{X|rrrrrrr}
    \rowcolor{gray!30}& Re & Amm & An & Pro & Prog & Ver & tot \\
    \hline
    Bonavigo Michele                        & 0 & 0 & 3 & 5 & 0 & 3 & 11 \\
    \rowcolor{gray!10}Casarotto Mattia      & 0 & 2 & 2 & 6 & 0 & 3 & 13 \\
    Massarenti Alessandro                   & 2 & 0 & 5 & 6 & 0 & 2 & 15 \\
    \rowcolor{gray!10}Peron Samuel          & 0 & 3 & 0 & 5 & 0 & 2 & 10 \\
    Pierobon Luca                           & 0 & 0 & 2 & 4 & 0 & 6 & 12 \\
    \rowcolor{gray!10}Romano Davide         & 2 & 2 & 0 & 6 & 0 & 2 & 12 \\
    Zarantonello Giorgio                    & 3 & 4 & 2 & 8 & 0 & 0 & 17 \\
    \hline                                  & 7 & 11 & 14 & 40 & 0 & 18 & 90 \\ 
    \end{tabularx}
\end{table}

La seguente tabella mostra il costo per ciascun ruolo:
\begin{table}[ht]
    \begin{tabularx}{\linewidth}{X|rrrrrrr}
    \rowcolor{gray!30}Ruolo & Ore & Costo \\
    \hline
    Responsabile                            & 7 & 210€ \\
    \rowcolor{gray!10}Amministratore        & 11 & 220€ \\
    Analista                                & 14 & 350€ \\
    \rowcolor{gray!10}Progettista           & 40 & 1000€ \\
    Programmatore                           & 0 & 0€ \\
    \rowcolor{gray!10}Verificatore          & 18 & 270€ \\
    \hline Totale                           & 90 & 2050€ \\ 
    \end{tabularx}
\end{table}

Grafico della distribuzione percentuale delle ore per ruolo:
\begin{center}
\begin{tikzpicture}
    \pie[
        color = {
            yellow!90!black,
            green!60!black,
            blue!60, red!70,
            cyan!60}
    ]{
        7.78/Responsabile,
        12.22/Amministratore,
        15.56/Analista,
        44.44/Progettista,
        20/Verificatore
    }
     
    \end{tikzpicture}
\end{center}



\section{Proof of Concept}

\textbf{Periodo di svolgimento}
\\ Data di inizio: 2023 - 02 - 06 \\ Data di fine: 2023 - 02 - 26

\textbf{Attività da svolgere}
    \begin{itemize}
        \item \textbf{Codifica Proof of Concept:} creazione di un'implementazione funzionante e dimostrativa di una tecnologia, prodotto o servizio per dimostrarne la fattibilità dell'idea e ne valida l'efficacia iniziale; 
        \item \textbf{Modifiche ai documenti:} vengono applicate revisioni a documenti esistenti, in cui vengono apportate correzioni, aggiornamenti, aggiunte o rimozioni di contenuti.
    \end{itemize}

\subsection{Ruoli attivi}
\begin{itemize}
    \item Responsabile 
    \item Amministratore 
    \item Analista 
    \item Progettista 
    \item Programmatore 
    \item Verificatore 
\end{itemize}

\subsection{Preventivo costi}

La seguente tabella mostra la suddivisione delle ore che ogni persona ricoprirà per ciascun ruolo:

\begin{table}[ht]
    \begin{tabularx}{\linewidth}{X|rrrrrrr}
    \rowcolor{gray!30}& Re & Amm & An & Pro & Prog & Ver & tot \\
    \hline
    Bonavigo Michele                        & 0 & 0 & 3 & 5 & 6 & 2 & 16 \\
    \rowcolor{gray!10}Casarotto Mattia      & 0 & 0 & 4 & 5 & 7 & 4 & 20 \\
    Massarenti Alessandro                   & 3 & 0 & 0 & 5 & 3 & 0 & 11 \\
    \rowcolor{gray!10}Peron Samuel          & 0 & 3 & 3 & 6 & 3 & 1 & 16 \\
    Pierobon Luca                           & 3 & 0 & 0 & 6 & 0 & 0 & 9  \\
    \rowcolor{gray!10}Romano Davide         & 2 & 2 & 0 & 5 & 4 & 1 & 14 \\
    Zarantonello Giorgio                    & 0 & 0 & 0 & 3 & 7 & 4 & 14 \\
    \hline                                  & 8 & 5 & 10 & 35 & 30 & 12 & 100 \\ 
    \end{tabularx}
\end{table}

La seguente tabella mostra il costo per ciascun ruolo:
\begin{table}[ht]
    \begin{tabularx}{\linewidth}{X|rrrrrrr}
    \rowcolor{gray!30}Ruolo & Ore & Costo \\
    \hline
    Responsabile                            & 8 & 240€ \\
    \rowcolor{gray!10}Amministratore        & 5 & 100€ \\
    Analista                                & 10 & 250€ \\
    \rowcolor{gray!10}Progettista           & 35 & 875€ \\
    Programmatore                           & 30 & 450€ \\
    \rowcolor{gray!10}Verificatore          & 12 & 180€ \\
    \hline Totale                           & 100 & 2095€ \\ 
    \end{tabularx}
\end{table}

Grafico della distribuzione percentuale delle ore per ruolo:
\begin{center}
\begin{tikzpicture}
    \pie[
        color = {
            yellow!90!black,
            green!60!black,
            blue!60, red!70,
            cyan!60, magenta!60}
    ]{
        8/Responsabile,
        5/Amministratore,
        10/Analista,
        35/Progettista,
        30/Programmatore,
        12/Verificatore
    }
     
    \end{tikzpicture}
\end{center}



\section{Progettazione e codifica della soluzione ai requisiti obbligatori}

Fase non ancora terminata.

\section{Progettazione e codifica requisiti opzionali}

\textbf{Periodo di svolgimento}
\\ Data di inizio: 2023 - 03 - 27 \\ Data di fine: 2023 - 04 - 09

\textbf{Attività da svolgere}
    \begin{itemize}
        \item \textbf{progettazione requisiti opzionali:} studio e ricerca di requisiti opzionali e come implementarali nel prodotto finale; 
        \item \textbf{codifica requisiti:} sviluppo di codice per implementare i requisiti opzionali;
        \item \textbf{test:} creazione e utilizzo di test per verificare il corretto funzionamento del codice sviluppato;
        \item \textbf{modifica documenti:} vengono applicate revisioni a documenti esistenti, in cui vengono apportate correzioni, aggiornamenti, aggiunte o rimozioni di contenuti.
    \end{itemize}

\subsection{Ruoli attivi}
\begin{itemize}
    \item Responsabile 
    \item Amministratore 
    \item Analista 
    \item Progettista 
    \item Programmatore 
    \item Verificatore 
\end{itemize}

\subsection{Preventivo costi}

La seguente tabella mostra la suddivisione delle ore che ogni persona ricoprirà per ciascun ruolo:

\begin{table}[ht]
    \begin{tabularx}{\linewidth}{X|rrrrrrr}
    \rowcolor{gray!30}& Re & Amm & An & Pro & Prog & Ver & tot \\
    \hline
    Bonavigo Michele                        & 1 & 0 & 0 & 0 & 5 & 2 & 8 \\
    \rowcolor{gray!10}Casarotto Mattia      & 0 & 3 & 0 & 4 & 5 & 0 & 12 \\
    Massarenti Alessandro                   & 0 & 0 & 0 & 3 & 5 & 5 & 13 \\
    \rowcolor{gray!10}Peron Samuel          & 3 & 0 & 2 & 3 & 5 & 0 & 13 \\
    Pierobon Luca                           & 1 & 1 & 3 & 0 & 5 & 0 & 10 \\
    \rowcolor{gray!10}Romano Davide         & 0 & 2 & 3 & 3 & 5 & 0 & 13 \\
    Zarantonello Giorgio                    & 0 & 2 & 4 & 2 & 5 & 0 & 13 \\
    \hline                                  & 5 & 8 & 12 & 15 & 35 & 7 & 82 \\ 
    \end{tabularx}
\end{table}

La seguente tabella mostra il costo per ciascun ruolo:
\begin{table}[ht]
    \begin{tabularx}{\linewidth}{X|rrrrrrr}
    \rowcolor{gray!30}Ruolo & Ore & Costo \\
    \hline
    Responsabile                            & 5  & 150€ \\
    \rowcolor{gray!10}Amministratore        & 8  & 160€ \\
    Analista                                & 12  & 300€ \\
    \rowcolor{gray!10}Progettista           & 15  & 375€ \\
    Programmatore                           & 35  & 525€ \\
    \rowcolor{gray!10}Verificatore          & 7  & 105€ \\
    \hline Totale                           & 82  & 1615€ \\ 
    \end{tabularx}
\end{table}

Grafico della distribuzione percentuale delle ore per ruolo:
\begin{center}
\begin{tikzpicture}
    \pie[
        color = {
            yellow!90!black,
            green!60!black,
            blue!60, red!70,
            cyan!60, magenta!60}
    ]{
        6.10/Responsabile,
        9.76/Amministratore,
        14.63/Analista,
        18.29/Progettista,
        42.68/Programmatore,
        8.54/Verificatore
    }
     
    \end{tikzpicture}
\end{center}



\section{Verifica e collaudo}


La seguente tabella mostra le ore che ogni persona ha ricoperto per ciascun ruolo:
\begin{table}[ht]
    \begin{tabularx}{\linewidth}{X|rrrrrrr}
    \rowcolor{gray!30}& Re & Amm & An & Pro & Prog & Ver & tot \\
    \hline
    Bonavigo Michele                        & 3(+1)    & 0          & 0       & 0     & 2       & 2       & 7(+1) \\

    \rowcolor{gray!10}Casarotto Mattia      & 0        & 0          & 0       & 0     & 2(+2)   & 4(-1)   & 6(+1) \\

    Massarenti Alessandro                   & 1        & 0          & 0       & 0     & 1(-1)   & 2       & 4(-1) \\ 

    \rowcolor{gray!10}Peron Samuel          & 0(-1)    & 0          & 0       & 0     & 0       & 4       & 4(-1) \\ 

    Pierobon Luca                           & 0(-1)    & 3          & 0       & 0     & 0       & 1       & 4(-1) \\ 

    \rowcolor{gray!10}Romano Davide         & 2(+1)    & 0          & 0       & 0     & 1       & 3       & 6(+1)\\

    Zarantonello Giorgio                    & 0        & 2          & 0       & 0     & 0       & 3       & 5\\

    \hline                                  & 6        & 5          & 0       & 0     & 6(+1)   & 19(-1)  & 36\\  
    \end{tabularx}
\end{table}

La seguente tabella mostra il costo per ciascun ruolo:
\begin{table}[ht]
    \begin{tabularx}{\linewidth}{X|rrrrrrr}
    \rowcolor{gray!30}Ruolo & Ore & Costo \\
    \hline
    Responsabile                            & 6    & 180€\\
    \rowcolor{gray!10}Amministratore        & 5    & 100€  \\
    Analista                                & 0    & 0\\
    \rowcolor{gray!10}Progettista           & 0    & 0\\
    Programmatore                           & 6    & 90€ (+15) \\
    \rowcolor{gray!10}Verificatore          & 19   & 300€ (-15€) \\
    \hline Totale                           & 37   & 655€\\ 
    \end{tabularx}
\end{table}

\subsection{Motivazione delle variazioni}
Ci sono state piccole variazioni rispetto a quanto preventivato in quanto si è dovuto dare più spazio alla verifica e al controllo del codice prodotto.

\subsection{Bilancio finale}
Il bilancio finale è stato di 655€, come preventivato. Anche se il bilancio è equivalente a quanto preventivato differiscono l'ammontare ore usate e la loro distribuzione nei ruoli.
