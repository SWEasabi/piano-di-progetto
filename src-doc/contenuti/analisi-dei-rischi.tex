\chapter{Analisi dei rischi}
Nel corso dello sviluppo del progetto è naturale incontrare vari tipi di problematiche, che con un'attenta e continua analisi dei rischi possono essere mitigate. Il piano per la gestione dei rischi viene suddiviso in 4 attività:
\begin{itemize}
    \item individuazione dei possibili eventi che possono portare a dei problemi durante l'avanzamento;
    \item analisi del problema, in particolare la probabilità con cui si possa verificare e le conseguenze
    negative che comporta;
    \item pianificazione di misure da prendere per impedire il verificarsi dei rischi e comportamenti da seguire nel caso in cui essi dovessero presentarsi. In questo modo si evita che un rischio possa diventare insostenibile;
    \item monitoraggio continuo dei rischi, cercando di prevenirli o minimizzando l'effetto negativo di quest'ultimi.
\end{itemize}

\section{Rischi tecnologici}

\begin{center}
    \begin{tabularx}{0.8\linewidth}{l|l}
        \multicolumn{2}{c}{\textbf{Problemi Hardware}}                                                          \\
        \hline{\textbf{Descrizione}}    & I membri del gruppo dispongono di diversi dispositivi i quali         \\ 
                                        & possono essere soggetti a guasti o problemi                           \\
        \textbf{Conseguenze}            & Ritardi non previsti nell'avanzamento del progetto                    \\
        \textbf{Incidenza}              & Bassa                                                                 \\
        \textbf{Pericolosità}           & Medio-Bassa                                                           \\
        \textbf{Precauzioni}            & Utilizzo di sistemi di backup automatici e remoti                     \\
        \textbf{Alternative}            & Laboratori messi a disposizione dall'ateneo, eventuali sostituzioni   \\ 
                                        & o riparazioni dei dispositivi danneggiati                             \\     
    \end{tabularx}
\end{center}

\begin{center}
    \begin{tabularx}{0.8\linewidth}{l|l}
        \multicolumn{2}{c}{\textbf{Problemi Software}}                                                          \\
        \hline{\textbf{Descrizione}}    & Il progetto verrà svilupato tramite l'utilizzo di vari software i     \\ 
                                        & quali possono presentare problemi o bug                               \\
        \textbf{Conseguenze}            & Ritardi nello sviluppo, perdita di dati e/o del lavoro svolto         \\
        \textbf{Incidenza}              & Bassa                                                                 \\
        \textbf{Pericolosità}           & Bassa                                                                 \\
        \textbf{Precauzioni}            & Sistemi di backup con possibilità di rollback e recupero di dati      \\
        \textbf{Alternative}            & Software sostitutivi con funzionalità e caratteristiche simili        \\ 
                                        & scelti in accordo con l'azienda                                       \\     
    \end{tabularx}
\end{center}

\begin{center}
    \begin{tabularx}{0.8\linewidth}{l|l}
        \multicolumn{2}{c}{\textbf{Problemi kit ambiente reale}}                                                \\
        \hline{\textbf{Descrizione}}    & Per simulare l'ambiente reale il progetto fa uso di kit               \\ 
                                        & comprendenti Raspberry e strisce led soggetti a                       \\ 
                                        & malfunzionamenti o indisponibilità                                    \\
        \textbf{Conseguenze}            & Possibili ritardi nel testing del prodotto                            \\
        \textbf{Incidenza}              & Media                                                                 \\
        \textbf{Pericolosità}           & Media                                                                 \\
        \textbf{Precauzioni}            & Il responsabile si tiene in contatto con l'azienda                    \\ 
                                        & per non trovarsi impreparati                                          \\
        \textbf{Alternative}            & L'azienda fornisce delle API simulate per l'accensione,               \\ 
                                        & regolazione e spegnimento di un "lampione" virtuale.                  \\ 
                                        & Inoltre l'azienda ha messo a disposizione delle macchine              \\ 
                                        & virtuali in cui lanciare server e simulatori                          \\      
    \end{tabularx}
\end{center}

\section{Rischi personali}

\begin{center}
    \begin{tabularx}{0.8\linewidth}{l|l}
        \multicolumn{2}{c}{\textbf{Problemi Decisionali}}                                                       \\
        \hline{\textbf{Descrizione}}    & L'azienda lascia libera la scelta delle tecnologie da                 \\ 
                                        & utilizzare e ciò porta a delle divergenze interne al gruppo           \\ 
        \textbf{Conseguenze}            & Dissenso all'interno del gruppo e peggioramento                       \\
        \textbf{Incidenza}              & Bassa                                                                 \\
        \textbf{Pericolosità}           & Medio-Bassa                                                           \\
        \textbf{Precauzioni}            & Utilizzo di strumenti e metodologie democratiche per le               \\ 
                                        & decisioni in questione                                                \\
        \textbf{Alternative}            & Il responsabile gestisce in modo appropriato le decisioni e le        \\ 
                                        & divergenze interne al gruppo                                          \\     
    \end{tabularx}
\end{center}

\begin{center}
    \begin{tabularx}{0.8\linewidth}{l|l}
        \multicolumn{2}{c}{\textbf{Mancanza di familiarità con la tecnologia}}                                  \\
        \hline{\textbf{Descrizione}}    & I membri del gruppo hanno livelli di conoscenza                       \\
                                        & e familiarità diversi con le tecnologie utilizzate                    \\
        \textbf{Conseguenze}            & Ogni membro ha tempistiche differenti per fare esperienza             \\                              
                                        & nell'utilizzo delle tecnologie che vengono usate                      \\
        \textbf{Incidenza}              & Alta                                                                  \\
        \textbf{Pericolosità}           & Media                                                                 \\
        \textbf{Precauzioni}            & I membri più ferrati in tali tecnologie aiuteranno i membri           \\ 
                                        & più carenti                                                           \\
        \textbf{Alternative}            & Ridistribuzione dei compiti assegnati                                 \\ 
    \end{tabularx}
\end{center}

\begin{center}
    \begin{tabularx}{0.8\linewidth}{l|l}
        \multicolumn{2}{c}{\textbf{Disponibilità dei membri}}                                                   \\
        \hline{\textbf{Descrizione}}    & I membri del gruppo hanno impegni extra-curricolari e lavorativi      \\
                                        & che non permettono di dedicarsi a tempo pieno nel progetto            \\
        \textbf{Conseguenze}            & Ritardi nell'avanzamento del lavoro individuale e di progetto         \\
        \textbf{Incidenza}              & Media                                                                 \\
        \textbf{Pericolosità}           & Media                                                                 \\
        \textbf{Precauzioni}            & Ogni membro si impegna a comunicare il prima possibile eventuali      \\ 
                                        & imprevisti o impegni al fine di una miglior organizzazione            \\
        \textbf{Alternative}            & Ridistribuzione dei compiti assegnati in caso di                      \\ 
                                        & indisponibilità persistenti                                           \\ 
    \end{tabularx}
\end{center}

\begin{center}
    \begin{tabularx}{0.8\linewidth}{l|l}
        \multicolumn{2}{c}{\textbf{Problemi Comunicativi}}                                                      \\
        \hline{\textbf{Descrizione}}    & La maggior parte delle comuncazioni avvengono telematicamente         \\
                                        & sia tra gruppo e committente sia tra i membri e quindi sono           \\
                                        & soggette a possibli problemi                                          \\                                 
        \textbf{Conseguenze}            & Ritardi nell'avanzamento del lavoro individuale e di progetto         \\
        \textbf{Incidenza}              & Bassa                                                                 \\
        \textbf{Pericolosità}           & Alta                                                                  \\
        \textbf{Precauzioni}            & Ogni membro si avvale di molteplici strumenti di supporto             \\
                                        & per evitare problemi di comunicazione                                 \\
        \textbf{Alternative}            & Il resposabile si avvarrà di strumenti alternativi                    \\ 
                                        & per la comunicazione                                                  \\ 
    \end{tabularx}
\end{center}


\section{Rischi organizzativi}

\begin{center}
    \begin{tabularx}{0.8\linewidth}{l|l}
        \multicolumn{2}{c}{\textbf{Gestione tempistiche}}                                                      \\
        \hline{\textbf{Descrizione}}    & Ogni membro del gruppo potrebbe avere problemi o imprevisti che      \\
                                        & non consentono di rispettare le scadenze prefissate                  \\                                 
        \textbf{Conseguenze}            & Ritardi nell'avanzamento del lavoro individuale e di progetto        \\
        \textbf{Incidenza}              & Media                                                                \\
        \textbf{Pericolosità}           & Alta                                                                 \\
        \textbf{Precauzioni}            & Ogni membro discute con il gruppo eventuali problemi o necessità     \\
                                        & che verranno prese in considerazione nell'assegnazione dei compiti   \\
                                        & e nella scelta delle scadenze                                        \\
        \textbf{Alternative}            & Il resposabile riassegna e distribuisce i compiti in modo da         \\ 
                                        & diminuire o rendere nulli eventuali ritardi                          \\ 
    \end{tabularx}
\end{center}


\begin{center}
    \begin{tabularx}{0.8\linewidth}{l|l}
        \multicolumn{2}{c}{\textbf{Stima costi}}                                                               \\
        \hline{\textbf{Descrizione}}    & Durante la stesura del preventivo dei costi il calcolo potrebbe      \\
                                        & risultare scorretto o non coerente a causa della poca esperienza     \\
                                        & del gruppo                                                           \\                                 
        \textbf{Conseguenze}            & Avanzamento di tempo utile e conseguente spreco di denaro in         \\
                                        & caso di sovrastima o ritardo nella consegna in caso di sottostima    \\
        \textbf{Incidenza}              & Media                                                                \\
        \textbf{Pericolosità}           & Alta                                                                 \\
        \textbf{Precauzioni}            & Ogni membro si impegna a seguire nel miglior modo possibile le       \\
                                        & decisioni prese nella fase di pianificazione del progetto            \\
        \textbf{Alternative}            & In caso di sovrastima il gruppo potrà utilizzare il tempo            \\
                                        & disponibile per migliorare o rifinire parti del progetto,            \\ 
                                        & in caso di sottostima il responsabile dovrà gestire i compiti        \\
                                        & e i membri in modo da poter rispettare al meglio le scadenze         \\ 
    \end{tabularx}
\end{center}


\section{Rischi legati ai requisiti}

\begin{center}
    \begin{tabularx}{0.8\linewidth}{l|l}
        \multicolumn{2}{c}{\textbf{Problemi nell'analisi dei requisiti}}                                       \\
        \hline{\textbf{Descrizione}}    & Possono esserci incomprensioni tra gli analisti e/o con l'azienda    \\
                                        & con possibilità di impiegare pìù ore di quelle preventivate,         \\
                                        & oppure alcuni requisiti non sono stati soddisfatti                   \\
                                        & (anche solo in parte) 
                                        \\                                 
        \textbf{Conseguenze}            & Aumento dei costi e ritardi nella consegna                           \\
        \textbf{Incidenza}              & Media                                                                \\
        \textbf{Pericolosità}           & Alta                                                                 \\
        \textbf{Precauzioni}            & Il responsabile si imepegna a risolvere eventuali dubbi tra gli      \\
                                        & analisti e/o con l'azienda, mentre i verificatori si impegnano a     \\
                                        & controllare nel dettaglio che i requisiti siano tutti soddisfatti    \\
        \textbf{Alternative}            & Il resposabile riassegna e distribuisce i compiti in modo da         \\ 
                                        & diminuire o rendere nulli eventuali ritardi                          \\ 
    \end{tabularx}
\end{center}


\begin{center}
    \begin{tabularx}{0.8\linewidth}{l|l}
        \multicolumn{2}{c}{\textbf{Stima costi}}                                                               \\
        \hline{\textbf{Descrizione}}    & In fase di sviluppo del progetto l'azienda potrebbe decidere di      \\
                                        & apportare modifiche ai requisiti e/o aggiungendone                   \\
                                        & altri obbligatori                                                    \\                                 
        \textbf{Conseguenze}            & Aumento delle ore utilizzate in fase di analisi con conseguente      \\
                                        & aumento dei costi e ritardi nella consegna                           \\
        \textbf{Incidenza}              & Bassa                                                                \\
        \textbf{Pericolosità}           & Alta                                                                 \\
        \textbf{Precauzioni}            & Come per gli incontri interni, anche gli incontri con l'azienda      \\
                                        & avranno un verbale in cui sarà scritto tutto il necessario           \\
                                        & al fine di rendere tempestivamente note eventuali modifiche          \\
        \textbf{Alternative}            & Il responsabile dovrà gestire i compiti                              \\
                                        & e i membri in modo da poter rispettare al meglio le scadenze         \\ 
    \end{tabularx}
\end{center}
