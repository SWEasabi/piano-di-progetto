\section{Progettazione e codifica requisiti obbligatori}

\textbf{Periodo di svolgimento}
\\ Data di inizio: 2023 - 02 - 28 \\ Data di fine: 2023 - 03 - 26

\textbf{Attività da svolgere}
    \begin{itemize}
        \item \textbf{Diagrammi di classe:} rappresentazione grafica del modello di classi e delle relazioni tra di esse in un software,mostrano le classi, i loro attributi e i loro metodi, così come le relazioni tra di esse, come l'ereditarietà, l'associazione, la composizione e l'aggregazione. Inoltre, i diagrammi di classe possono mostrare le interfacce delle classi e le relazioni tra le interfacce; 
        \item \textbf{diagrammi di sequenza:} descrivono l'interazione tra gli oggetti all'interno di un software in un determinato scenario di utilizzo e mostrano l'ordine temporale delle operazioni e delle chiamate tra gli oggetti potendo visualizzare come essi collaborano per raggiungere uno specifico obiettivo ;
        \item \textbf{diagrammi di attività:} descrivono il flusso di lavoro di un processo o di una procedura all'interno del software e mostrano un insieme di azioni, decisioni e attività che sono necessarie per raggiungere uno specifico obiettivo all'interno del sistema;
        \item \textbf{sviluppo codice:} sviluppo del codice per la realizzazione del software ichiesto proffisto dei requisiti obbligatori imposti dal committente;
        \item \textbf{test:} creazione e utilizzo di test per verificare il corretto funzionamento del codice sviluppato;
        \item \textbf{manuale utente:} manuale che descrive l'utilizzo del software utile per l'utente finale.
    \end{itemize}

\subsection{Ruoli attivi}
\begin{itemize}
    \item Responsabile 
    \item Amministratore
    \item Analista
    \item Progettista 
    \item Programmatore 
    \item Verificatore 
\end{itemize}

\subsection{Preventivo costi}

La seguente tabella mostra la suddivisione delle ore che ogni persona ricoprirà per ciascun ruolo:

\begin{table}[ht]
    \begin{tabularx}{\linewidth}{X|rrrrrrr}
    \rowcolor{gray!30}& Re & Amm & An & Pro & Prog & Ver & tot \\
    \hline
    Bonavigo Michele                        & 0 & 2 & 3 & 7 & 8 & 3 & 23 \\
    \rowcolor{gray!10}Casarotto Mattia      & 0 & 1 & 5 & 4 & 6 & 1 & 17 \\
    Massarenti Alessandro                   & 0 & 0 & 1 & 8 & 9 & 2 & 20 \\
    \rowcolor{gray!10}Peron Samuel          & 0 & 0 & 3 & 7 & 10 & 3 & 23 \\
    Pierobon Luca                           & 5 & 0 & 1 & 7 & 12 & 0 & 25 \\
    \rowcolor{gray!10}Romano Davide         & 5 & 3 & 2 & 6 & 9 & 2 & 27 \\
    Zarantonello Giorgio                    & 5 & 2 & 0 & 7 & 6 & 4 & 24 \\
    \hline                                  & 15 & 8 & 15 & 46 & 60 & 15 & 159 \\ 
    \end{tabularx}
\end{table}

La seguente tabella mostra il costo per ciascun ruolo:
\begin{table}[ht]
    \begin{tabularx}{\linewidth}{X|rrrrrrr}
    \rowcolor{gray!30}Ruolo & Ore & Costo \\
    \hline
    Responsabile                            & 15  & 450€ \\
    \rowcolor{gray!10}Amministratore        & 8 & 160€ \\
    Analista                                & 15  & 375€ \\
    \rowcolor{gray!10}Progettista           & 46  & 1125€ \\
    Programmatore                           & 60 & 900€ \\
    \rowcolor{gray!10}Verificatore          & 15  & 225€ \\
    \hline Totale                           & 151  & 3235€ \\ 
    \end{tabularx}
\end{table}

Grafico della distribuzione percentuale delle ore per ruolo:
\begin{center}
\begin{tikzpicture}
    \pie[color = {
        yellow!90!black,
        green!60!black,
        blue!60, red!70,
        cyan!60, magenta!60}
    ]{
        9.43/Responsabile,
        5.03/Amministratore,
        9.43/Analista,
        28.93/Progettista,
        37.74/Programmatore,
        9.43/Verificatore
    }
     
    \end{tikzpicture}
\end{center}

