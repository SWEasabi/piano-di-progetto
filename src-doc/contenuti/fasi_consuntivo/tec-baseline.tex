\section{Technology Baseline}

La seguente tabella mostra le ore che ogni persona ha ricoperto per ciascun ruolo:

\begin{table}[ht]
    \begin{tabularx}{\linewidth}{X|rrrrrrr}
    \rowcolor{gray!30}& Re & Amm & An & Pro & Prog & Ver & tot \\
    \hline
    Bonavigo Michele                        & 2 (+2)     & 1 (+1)   & 4         & 6 (+1)    & 0     & 2         & 15 (+4)\\ 

    \rowcolor{gray!10}Casarotto Mattia      & 0          & 3 (+1)   & 4         & 5 (+2)    & 0     & 1         & 13 (+3)\\ 

    Massarenti Alessandro                   & 5 (+1)     & 0        & 4 (+2)    & 4         & 0     & 3 (+3)    & 16 (+6) \\ 

    \rowcolor{gray!10}Peron Samuel          & 0          & 2        & 3         & 2 (-1)    & 0     & 0 (-1)    & 7 (-2) \\  

    Pierobon Luca                           & 4          & 1 (+1)   & 1         & 3         & 0     & 3         & 12 (+1) \\ 

    \rowcolor{gray!10}Romano Davide         & 1 (-2)     & 0        & 1         & 2         & 0     & 3         & 7 (-2) \\ 

    Zarantonello Giorgio                    & 3          & 0        & 2         & 5         & 0     & 0         & 10     \\ 

    \hline                                  & 15 (+1)    & 7 (+3)   & 19 (+2)   & 27 (+2)   & 0     & 12 (+2)   & 80 (+10) \\ 
    \end{tabularx}
\end{table}


La seguente tabella mostra il costo per ciascun ruolo:
\begin{table}[ht]
    \begin{tabularx}{\linewidth}{X|rrrrrrr}
    \rowcolor{gray!30}Ruolo & Ore & Costo \\
    \hline
    Responsabile                            & 15   & 450€ (+30€)\\
    \rowcolor{gray!10}Amministratore        & 7    & 140€ (+60€)\\
    Analista                                & 19   & 475€ (+50€)\\
    \rowcolor{gray!10}Progettista           & 27   & 675€ (+50€)\\
    Programmatore                           & 0    & 0€ \\
    \rowcolor{gray!10}Verificatore          & 12   & 180€ (+30€) \\
    \hline Totale                           & 75   & 1920€ (+220€) \\ 
    \end{tabularx}
\end{table}

\subsection{Motivazione delle variazioni}
Sono state utilizzate più ore rispetto a quelle preventivate, in quanto i problemi riscontrati nella fase precedente hanno portato ad un avanzo di ore che sono state in parte riutilizzate in questa fase.

\subsection{Bilancio finale}
Il bilancio finale è stato di 1920€, il costo totale è aumentato di 220€ rispetto al bilancio iniziale. Gli obiettivi sono stati raggiunti ma sono state utilizzate più risorse (tempo-denaro) rispetto a quanto preventivato, il bilancio finale è negativo ma non si rende necessaria una ripianificazione dei prossimi periodi.
