\chapter{Introduzione}

\section{Scopo del Documento}
Nel seguente documento viene illustrato un prospetto di pianificazione in modo dettagliato e delle modalità attraverso le quali avverrà lo sviluppo del progetto. Il documento tratterà, in ordine, i seguenti punti:
\begin{itemize}
    \item analisi dei rischi;
    \item descrizione del modello di sviluppo adottato;
    \item suddivisione delle varie fasi con conseguente assegnazione dei ruoli;
    \item stima dei costi e delle risorse necessarie.
\end{itemize}

\section{Scopo del prodotto}
L'obiettivo di SWEasabi e dell'azienda ImolaInformatica S.p.A. è lo sviluppo di un sistema per l'ottimizzazione dell'illuminazione pubblica che permetta ai gestori di sfruttare le possibilità dell'IoT per fornire il servizio di illuminazione pubblica avendo a cuore l'aspetto green. Un sistema così congegnato consentirebbe, da un lato, di garantire sicurezza stradale e sociale, e dall'altro permetterebbe di risparmiare energia, dunque, risorse economiche e ambientali.


\section{Glossario}
Per evitare ambiguità relative alle terminologie utilizzate è stato creato un documento denominato "Glossario". Questo documento comprende tutti i termini tecnici scelti dai membri del gruppo e utilizzati nei vari documenti con le relative definizioni. Tutti i termini inclusi in questo glossario, vengono segnalati all'interno del documento con l'apice accanto alla parola.

\section{Riferimenti}
\subsection{Riferimenti Normativi}
\begin{itemize}
    \item \textit{Norme-di-progetto};
    \item \href{https://www.math.unipd.it/~tullio/IS-1/2022/Progetto/C2.pdf}{capitolato d'appalto C2}.
\end{itemize}

\subsection{Riferimenti Informativi}
\begin{itemize}
    \item \textit{Piano-di-Qualifica};
    \item \href{https://www.math.unipd.it/~tullio/IS-1/2022/Dispense/T03.pdf}{Il ciclo di vita del software} - Materiale didattico del corso di Ingegneria del Software:
    \begin{itemize}
        \item modello incrementale - Slides: 19, 20, 21 e 22.
    \end{itemize}
\end{itemize}
